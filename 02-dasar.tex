\chapter{Dasar-Dasar}

\section{Tipe Data}

Sebagaimana kebanyakan bahasa pemrograman, mengenali tipe data adalah hal yang penting untuk diketaui, karena perbedaan tipe data dapat menyebabkan perbedaan operasi yang dapat dilakukan kepadanya.

Tipe data secara garis besar dapat dikelompokan menjadi : angka, karakter, \textit{boolean}, dan larik.

\subsection{Angka}

Tipe data untuk angka di Kotlin mirip dengan Java. Dan untuk karakter bukan dianggap sebagai angka di Kotlin. Berikut adalah tipe data angka yang dapat digunakan beserta ukurannya :

\begin{center}
\begin{tabular}{|l|l|}
	\hline
	\textbf{Tipe} & \textbf{Panjang Bit} \\
	\hline
	\hline
	Double & 64 \\
	\hline
	Float & 32 \\
	\hline
	Long & 64 \\
	\hline
	Int & 32 \\
	\hline
	Short & 16 \\
	\hline
	Byte & 8 \\
	\hline
\end{tabular}
\end{center}

\begin{itemize}
	\item \textbf{Format Angka}
	
	Format angka pada Kotlin dapat mengakomodir beberapa format berikut :
	
	\begin{itemize}
		\item Desimal, contohnya adalah \texttt{432}, untuk tipe data Long dituliskan sebagai \texttt{432L}.
		\item Hexadesimal, contohnya adalah \texttt{0xa4}
		\item Biner, contohnya adalah \texttt{0b00111100}
	\end{itemize}
	
	Sebagai catatan bahwa tipe format oktal tidak didukung di Kotlin. Kotlin juga mendukung bilangan pecahan sebagai berikut :
	
	\begin{itemize}
		\item Double, contohnya adalah 12.34
		\item Float, contohnya adalah 154.3f
	\end{itemize}
	
	\item \textbf{Garis Bawah Pada Format Angka}
	
	Kita dapat menggunakan garis bawah sebagai tanda pada angka yang kita isikan sebagai pengganti digit atau format lain. Ini didukung oleh Kotlin versi 1.1 ke atas, artinya bila masih menggunakan Kotlin versi sebelumnya, kode angka yang dibangun dengan garis bawah akan berantakan. Berikut contoh penulisan angka dengan garis bawah :
	
	\begin{lstlisting}
val satuJuta = 1_000_000
val telp = 0821_3828_3607
	\end{lstlisting}
		
	\item \textbf{Konversi Eksplisit}
	
	Konversi eksplisit diperlukan karena tipe yang lebih kecil bukan berarti merupakan sub-tipe yang lebih besar, contohnya adalah seperti kode berikut :
	
	\begin{lstlisting}
val a: Int? = 1
val b: Long? = a
print(a == b)
	\end{lstlisting}
	
	Hasil dari kode tersebut, pada saat kompilasi akan menghasilkan 2 (dua) kesalahan yaitu, yang pertama, variabel dengan tipe data \texttt{Int} tidak dapat dimasukkan langsung isinya ke dalam variabel dengan tipe data \texttt{Long}.
	
	Untuk tujuan ini, kita perlu melakukan konversi secara eksplisit dengan beberapa fungsi berikut, dan tiap variabel berjenis angka memilikinya :
	
	\begin{itemize}
		\item \texttt{toByte(): Byte}
		\item \texttt{toShort(): Short}
		\item \texttt{toInt(): Int}
		\item \texttt{toLong(): Long}
		\item \texttt{toFloat(): Float}
		\item \texttt{toDouble(): Double}
		\item \texttt{toChar(): Char}
	\end{itemize}
	
	Contoh implementasi dari kode sebelumnya akan menjadi seperti ini :
	
	\begin{lstlisting}
val a: Int = 1
val b: Long? = a.toLong()
print(a.toLong() == b)
	\end{lstlisting}
	
	\item \textbf{Operasi}
	
		Ada beberapa operasi, atau lebih dikenal sebagai fungsi, yang biasa dilakukan pada angka yang disediakan Kotlin, berikut adalah diantaranya :
		
		\begin{itemize}
			\item \texttt{shl(bits)}, geser bit ke kiri
			\item \texttt{shr(bits)}, geser bit ke kanan
			\item \texttt{ushr(bits)}, geser bit tanpa tanda ke kanan 
			\item \texttt{and(bits)}, operator bit dan
			\item \texttt{or(bits)}, operator bit atau
			\item \texttt{xor(bits)}, operator bit \textit{xor}
			\item \texttt{inv()}, operator bit inversi
		\end{itemize}
	
\end{itemize}

\subsection{Karakter}

Sama bahwa tipa data angka yang lebih kecil bukan berarti sub-tipe data angka yang lebih besar, berbeda dengan Java, dimana tipe karakter (\texttt{Char}) tidak dapat diperlakukan seperti tipe data angka. Berikut contoh kode yang tidak bisa dikompilasi :

\begin{lstlisting}
fun main(args: Array<String>) {
	val a: Char = 1
	print(a)
}
\end{lstlisting}

Hasil kode di atas tidak dapat dikompilasi karena terdapat kesalahan / \textit{error} pada baris ke-2. 

Tipe data karakter harus memiliki tanda kutip tunggal sepasang, dan karakter yang berada di tengahnya. Contoh kodenya adalah sebagai berikut :

\begin{lstlisting}
fun main(args: Array<String>) {
	val a: Char = '1'
	print(a)
}
\end{lstlisting}

Pada kode di atas, variabel \texttt{a} bertipe data \texttt{Char} dengan diberikan nilai berupa karakter \texttt{1}. Kemudian mencetak isi variabel \texttt{a} ke layar monitor.

Karakter-karakter tertentu dapat dimunculkan dengan menggunakan \textit{backslash}. Contoh kode untuk karakter-karakter tertentu atau spesial adalah sebagai berikut :

\begin{lstlisting}
fun main(args: Array<String>) {
	val a: Char = '\t'
	val b: Char = '\b'
	val c: Char = '\n'
	val d: Char = '\r'
	val e: Char = '\''
	val f: Char = '\"'
	val g: Char = '\\'
	val h: Char = '\$'
	
	println("isi a : $a text")
	println("isi b : $b text")
	println("isi c : $c text")
	println("isi d : $d text")
	println("isi e : $e text")
	println("isi f : $f text")
	println("isi g : $g text")
	println("isi h : $h text")
}
\end{lstlisting}

Hasil keluaran dari kode di atas adalah sebagai berikut :

\begin{lstlisting}
isi a : 	text
isi b : text
isi c : 
 text
 text
isi e : ' text
isi f : " text
isi g : \ text
isi h : $ text
\end{lstlisting}

Seperti pada pembahasan di awal bahwa tipe data \texttt{Char} bukan merupakan sub-tipe dari tipe data \texttt{Int}, tetapi sebetulnya bisa saja di konversi secara eksplisit, berikut adalah contoh kodenya :

\begin{lstlisting}
fun main(args: Array<String>) {
	val a: Int
	a = ubahKeDesimal('b')
	println("nilai a : $a")
}

fun ubahKeDesimal(c: Char): Int {
	return c.toInt()
}
\end{lstlisting}

Hasil dari kode tersebut adalah sebagai berikut :

\begin{lstlisting}
nilai a : 98
\end{lstlisting}

\subsection{\textit{Boolean}}

Tipe \textit{boolean} adalah tipe dengan nilai hanya ada 2 (dua) pilihan, \texttt{true} dan \texttt{false}. Contoh kodenya adalah sebagai berikut :

\begin{lstlisting}
fun main(args: Array<String>) {
	val a: Boolean = true
	println("nilai a = $a")
}
\end{lstlisting}

Nilai yang diisikan ke variabel \texttt{a} yang bertipe data \texttt{Boolean} adalah \texttt{true}, yang kemudian apabila di \textit{compile} dan di eksekusi akan menghasilkan keluaran sebagai berikut :

\begin{lstlisting}
nilai a = true
\end{lstlisting}

\subsection{Larik}

Larik dalam bahasa pemrograman Kotlin berbentuk sebuah kelas \texttt{Array}. Untuk membuat larik, kita dapat memanggil fungsi \texttt{arrayOf()} atau \texttt{arrayOfNulls()}, contoh kodenya adalah sebagai berikut :

\begin{lstlisting}
fun main(args: Array<String>) {
	val a: Array<Any>
	a = arrayOf('a', 'c', 'i')
	
	println("isi larik a : ")
	for(i in a) {
		print(" $i ")
	}
	
	println()
	val b: Array<Any?>
	b = arrayOfNulls(5)
	
	println("isi larik b : ")
	for(i in b) {
		print(" $i ")
	}
}
\end{lstlisting}

Kode tersebut pada baris ke-2 akan mendeklarasikan atau menyiapkan variabel \texttt{a} dengan bentuk \texttt{Array} yang mampu menampung tipe data apa pun, yang kebetulan pada baris ke-3 diisikan dengan data bertipe karakter. Baris ke-5 sampai dengan baris ke-8 digunakan untuk menampilkan isi dari larik \texttt{a}.

Pada baris ke-11, kita mendeklarasikan atau menyiapkan variabel \texttt{b} yang berbentuk larik dengan tipe data \texttt{Any} yang mampu menampung tipe data apapun, ada tanda berbentuk \texttt{?} setelahnya yang artinya data yang diisikan dapat berupa data \texttt{null}, karena memang pada baris ke-12 kita akan isikan larik \texttt{b} ini dengan 5 (lima) data \texttt{null}.

Kemudian baris ke-14 sampai dengan baris ke-17 digunakan untuk mencetak isi dari larik \texttt{b}.

Kita juga dapat membuat larik dengan menggunakan konstruktor-nya \texttt{Array}, yang menjadi parameter dari konstruktor ini adalah jumlah data dan nilai \textit{default} yang akan diisikan. Contoh kodenya adalah seperti berikut ini : 

\begin{lstlisting}
fun main(args: Array<String>) {
	val a: Array<Any>
	a = Array(3, { _ -> '-' })
	
	println("isi larik a :")
	for(i in a) {
		println(" $i ")
	}
}
\end{lstlisting}

Pada baris ke-3, larik akan disiapkan untuk menampung 3 (tiga) data sesuai dengan parameter pertama dari konstruktor ini, dengan nilai \textit{default} berupa tanda strip \texttt{-} sesuai parameter ke-2 dari konstruktor ini.

Kode dari parameter ke-2 yaitu \texttt{\{ \_ -> '-' \}} nampak membingungkan, 
sebetulnya formatnya adalah \texttt{\{ Int -> T \}} dengan \texttt{Int} dapat digantikan variabel apapun yang nantinya akan menampung nilai \textit{integer}, pada kode yang kita buat menggunakan tanda garis bawah (\texttt{\_}) karena nilainya tidak dibutuhkan. 

Kemudian ada huruf \texttt{T} disana yang dapat digantikan dengan data apapun, termasuk karakter yang kita masukan berupa tanda minus (\texttt{-}).

Kotlin juga menyediakan kelas larik untuk tipe data primitif tanpa harus mendeklarasikan dalam kurung siku, contohnya seperti kelas \texttt{ByteArray}, \texttt{ShortArray}, \texttt{IntArray}, dan sebagainya. Berikut adalah contoh kode untuk penerapan kelas larik tipe data primitif ini :

\begin{lstlisting}
fun main(args: Array<String>) {
	val a: IntArray
	a = intArrayOf(3, 2, 7)
	
	println("isi larik a:")
	for(i in a) {
		println(" $i ")
	}
}
\end{lstlisting}

Pada baris ke-2 adalah persiapan variabel \texttt{a} dengan tipe data \texttt{IntArray}. Pada baris ke-3, variabel atau larik \texttt{a} diisikan dengan 3 (tiga) angka yaitu \texttt{3}, \texttt{2}, dan \texttt{7}.

Pada baris ke-5 sampai baris ke-8 adalah kode untuk menampilkan isi dari larik \texttt{a}.

\subsection{\textit{String}}

Tipe data \textit{string} digunakan untuk menampung 1 (satu) karakter atau lebih, yang sebenarnya seperti sebuah larik untuk menampung banyak karakter.

Penggunaan tipe data \texttt{String} ini sama seperti bahasa pemrograman lain, terutama Java, dapat langsung diisikan kata atau kalimat yang menjadi isinya seperti contoh berikut :

\begin{lstlisting}
val kata = "Ini isi dari variabel kata"
\end{lstlisting}

Kode tersebut artinya memasukan kalimat \texttt{Ini isi dari variabel kata} ke variabel \texttt{kata}. Dalam \textit{string} kita juga dapat menambahkan \textit{escape character} seperti kode berikut :

\begin{lstlisting}
val kata = "Ini isi dari variabel kata\n"
\end{lstlisting}

Kode tersebut nantinya akan menjadikan kursor pindah ke baris berikutnya.

Pada Kotlin, ada yang namanya \textit{string} mentah, dimana pengetikan nilai \textit{string} ini tidak perlu menggunakan \textit{escape character}, dan setiap spasi atau baris baru akan dicetak sebagaimana adanya. berikut contoh kodenya :

\begin{lstlisting}
fun main(args: Array<String>) {
	val code = """
	  ini contoh raw string
	  spasi dan baris baru akan terbaca
	  sebagaimana mestinya
	  
	    ini saat pindah baris baru
	"""
	
	println(code)
}
\end{lstlisting}

Hasil keluaran dari kode di atas adalah sebagai berikut :

\begin{lstlisting}
        ini contoh raw string
        spasi dan baris baru akan terbaca
        sebagaimana mestinya
        
          ini saat pindah baris baru
\end{lstlisting}

Terlihat bahwa hasil keluaran dari kode tersebut, mirip dengan kondisi di kodenya selain spasi kiri yang mungkin terlalu lebar. Kita juga dapat menggunakan \textit{prefix} apapun sebagai tanda bahwa ini adalah awal dari pencetakan dengan deklarasi \texttt{trimMargin()}, contoh kodenya adalah sebagai berikut :

\begin{lstlisting}
fun main(args: Array<String>) {
    var kode = """
        |  ini contoh raw string
        |  spasi dan baris baru akan terbaca
        |  sebagaimana mestinya
        
        |    ini saat pindah baris baru
    """.trimMargin("|")
    
    println(kode)
}
\end{lstlisting}

Kita akan melihat perbedaan di baris ke-8, yaitu adanya \texttt{trimMargin("|")}, yang sebetulnya ini adalah fungsi milik \textit{raw string} untuk melakukan pemotongan baris awal, sedangkan tanda garis lurus (\texttt{|}) adalah tanda dimulainya baris paling kiri, tanda awal baris ini pun dapat diganti sesuai keinginan kita.

Pada \textit{string} di Kotlin, dapat menggunakan \textit{template} dengan tanda \texttt{\$} untuk menambahkan hasil dari ekspresi sebuah perintah ke dalam teks. Contoh kodenya adalah sebagai berikut :

\begin{lstlisting}
fun main(args: Array<String>) {
	val a = "tamami"
	
	println("isi dari a adalah $a")
	println("panjang dari a adalah ${a.length}")
}
\end{lstlisting}

Penggunaan \textit{template} ada pada baris ke-4 dan baris ke-5. Dimana pada baris ke-4 menampilkan isi dari variabel \texttt{a} dengan \textit{template} \texttt{\$a}, dan pada baris ke-5 memanggil fungsi \texttt{length} milik variabel \texttt{a} yang bertipe \textit{string}, pemanggilan fungsi \texttt{length} ini dilakukan dengan \textit{template} \texttt{\$\{a.length\}}.

\section{Paket}

Pada Kotlin mendukung penamaan paket seperti di Java, hanya saja nama paket tidak terikat dengan nama direktori tempat \textit{file} kode sumber berada, tetapi hasil dari kompilasi akan ditempatkan dalam direktori / \textit{folder} sesuai dengan nama paket, bila direktori / \textit{folder} belum ada, maka akan dibuatkan direktori / \textit{folder} baru dengan nama sama dengan nama paket. Berikut contoh deklarasi paket pada kode sumber :

\begin{lstlisting}
package tester

fun main(args: Array<String>) {
	println("ini dicetak dalam paket tester")
}
\end{lstlisting}

Deklarasinya ada pada baris ke-1. Walaupun penempatan kode sumber tidak terikat dengan nama paket, namun lebih baik bila kode sumber tetap disimpan dalam direktori / \textit{folder} sesuai dengan nama paket, agar memudahkan pencarian kelas.

Apabila suatu kelas akan menggunakan kelas lain dari paket yang lain, kita perlu mendeklarasikan dengan perintah \texttt{import}

\section{Mengatur Alur}

