%\documentclass{wileysev} % ini 7x10
\documentclass{wileysix} % ini 6x9

\usepackage{w-bookps} % postscript text

\usepackage{graphicx}
\usepackage{color}
\usepackage{fancyhdr}
\usepackage[english]{babel}
\usepackage{listings}
\usepackage{hyperref}
\usepackage{enumerate}

\hypersetup{
	colorlinks= true,
	linkcolor= blue,
	filecolor= magenta, 
	urlcolor= blue,
	bookmarks=true
}

\definecolor{gray}{cmyk}{0,0,0,0.5}

% level section head to appear 
% 0: no section number
% 1: section
% 2: subsection
% 3: subsubsection
\setcounter{secnumdepth}{3}

% level section head to appear in table of contents
% 0: chapter title
% 1: section titles
% 2: subsection titles
% 3: subsubsection titles
\setcounter{tocdepth}{2}

%\draft % double space between lines, current date and time printed at bottom of page
%\renewcommand{\arraystretch}{0.6} % tables in double space

\offprintinfo{Kotlin Siapa Suka, \\ Dasar-Dasar Pemrograman}{P. Tamami}

% list code program
\definecolor{codegreen}{rgb}{0,0.6,0}
\definecolor{codegray}{rgb}{0.5,0.5,0.5}
\definecolor{codepurple}{rgb}{0.58,0,0.82}
\definecolor{backcolor}{rgb}{0.95,0.95,0.92}

\lstdefinestyle{mystyle}{
  backgroundcolor=\color{backcolor},
  commentstyle=\color{codegreen},
  keywordstyle=\color{magenta},
  stringstyle=\color{codepurple},
  basicstyle=\footnotesize,
  breakatwhitespace=false,
  breaklines=true,
  captionpos=b,
  keepspaces=true,
  numbers=left,
  numbersep=5pt,
  showspaces=false,
  showstringspaces=false,
  showtabs=false,
  tabsize=2
}

\lstset{style=mystyle}
% end list code program

\begin{document}

\booktitle{Kotlin, Siapa Suka} 
\subtitle{Dasar}

\authors{P. Tamami\\ 
\affil{BPPKAD Kab. Brebes}}

\halftitlepage
 
\titlepage

\dedication{Untuk Istriku yang selalu memberi semangat, dan anak-anak yang selalu ceria}

\tableofcontents
\listoffigures
\listoftables

\begin{foreword}
Saat melihat keunggulan dari bahasa pemrograman Java yang mudah untuk di\textit{maintenance}, dapat berjalan di berbagai \textit{platform}, berorientasi objek, dan beberapa keunggulan lain, ada beberapa penyempurnaan yang dilakukan oleh bahasa pemrograman Kotlin, yang sama-sama berjalan di atas JVM. 

Dalam buku ini akan dijelaskan dasar dari pemrograman Kotlin yang menawarkan penulisan kode yang lebih ringkas, menjamin kesalahan seluruh kelas dari \textit{exception} \textbf{null}, dan yang tidak kalah penting adalah integrasinya dengan sistem yang dibangun dengan menggunakan bahasa Java.

Silahkan menikmati buku yang kurang dari sempurna ini, dan berharap penulis mendapatkan kritik yang membangun guna perubahan isi buku ini ke arah yang lebih sempurna.

\vspace{4mm}
4 Mei 2017

\vspace{5mm}
Penulis
\end{foreword}

\chapter{Memulai} 

Perlu diketahui bahwa Kotlin ini adalah bahasa pemrograman yang berjalan di atas JVM, sehingga diperlukan Java Runtime untuk menjalankannya.

Cara termudah untuk memasangkan atau meng\textit{install} \textit{compiler} Kotlin adalah dengan mengunduh di halaman \url{https://github.com/JetBrains/kotlin/releases/}, kemudian melakukan \textit{unzip} dan menambahkan direktori \texttt{bin} ke dalam \textit{path} sistem.

Untuk memastikan bahwa Kotlin sudah terpasang dan dapat digunakan, kita seharusnya dapat menjalankan perintah berikut di konsol pada Linux atau \textit{command prompt} milik Windows, berikut perintahnya :

\begin{lstlisting}
kotlinc -version
\end{lstlisting}

Perintah tersebut sebetulnya untuk mencetak informasi tentang versi \textit{compiler} Kotlin yang aktif. Dan seharusnya akan muncul informasi yang kurang lebih sebagai berikut :

\begin{lstlisting}
info: Kotlin Compiler version 1.1.2-2
\end{lstlisting}

Tentunya versi yang keluar akan berbeda tergantung apa yang kita \textit{install}.

Percobaan berikutnya adalah menampilkan versi \textit{runtime environment} dari Kotlin, jika perintah \texttt{kotlinc} digunakan untuk melakukan \textit{compile} (kompilasi) terhadap kode yang kita ketik / tulis menjadi bahasa biner, fungsi dari \textit{runtime environment} adalah menerjemahkan bahasa biner hasil \textit{compile} oleh \texttt{kotlinc} menjadi bahasa \textit{native} sesuai sistem operasi yang digunakan, inilah prinsip yang digunakan bahasa pemrograman Java yang tetap digunakan oleh Kotlin, karena memang Kotlin masih menggunakan JRE (\textit{Java Runtime Environment}).

Perintah untuk melihat versi \textit{runtime environment} dari Kotlin adalah sebagai berikut :

\begin{lstlisting}
kotlin -version
\end{lstlisting}

Dengan hasil keluaran di layar monitor seperti ini :

\begin{lstlisting}
Kotlin version 1.1.2-2 (JRE 1.8.0_121-b13)
\end{lstlisting}

Versi Kotlin seharusnya sama dengan versi \textit{compiler}-nya. Sedangkan muncul tambahan informasi \texttt{JRE 1.8.0\_121-b13}, inilah yang menunjukan bahwa Kotlin masih menggunakan JRE untuk menjalankan programnya, karena memang sebelum melakukan instalasi Kotlin, Java harus di\textit{install} terlebih dahulu.

\section{Katakan Hai}

Setelah melakukan percobaan dasar seperti di atas, kita akan mencoba menjalankan kode pertama yang kita buat dengan Kotlin. Berikut adalah langkahnya :

\begin{enumerate}[1.]
	\item Membuka editor teks seperti notepad, atom, notepad++, atau aplikasi sejenis. 
	\item Mengetikan kode berikut :
	
\begin{lstlisting}
fun main(args: Array<String>) {
    println("Hai, selamat datang")
}
\end{lstlisting}

	\item Simpanlah dengan nama apapun, berikan ekstensi \texttt{kt}, misal kita beri nama \textit{file} tersebut dengan \texttt{Test.kt}.
	\item Buka konsol atau \textit{command prompt} dan aktifkan ke direktori tempat kita simpan \textit{file} \texttt{Test.kt} tadi.
	\item \textit{Compile file} \texttt{Test.kt} tersebut dengan perintah berikut :
	
\begin{lstlisting}
kotlinc Test.kt
\end{lstlisting}

	\item Hasil dari \textit{compile} tersebut adalah berupa \textit{file} \texttt{TestKt.class}
	\item Untuk menjalankan hasil program yang telah kita \textit{compile}, gunakan perintah berikut :
	
\begin{lstlisting}
kotlin TestKt
\end{lstlisting}

	\item Kemudian akan program / aplikasi akan menghasilkan keluaran sebagai berikut :
	
\begin{lstlisting}
Hai, selamat datang
\end{lstlisting}

	\item Sampai titik ini, kita berhasil menjalankan kode yang telah kita buat.
\end{enumerate}

Jadi sebetulnya, untuk memulai koding dengan bahasa Kotlin cukup sederhana, tinggal siapkan \textit{berekstensi} \texttt{kt}, kemudian sertakan blok kode program berikut :

\begin{lstlisting}
fun main(args: Array<String>) {
	...
}
\end{lstlisting}

Seluruh program yang dibangun dengan Kotlin akan berawal dari fungsi \texttt{main} ini.

\section{Sintak Dasar}

\subsection{Deklarasi Paket}

Sama seperti bahasa pemrograman Java, deklarasi paket berada di awal kode seperti contoh berikut :

\begin{lstlisting}
package nama.paket

import java.net.*
...
\end{lstlisting}

Perbedaannya adalah bahwa nama paket tidak perlu disesuaikan atau disamakan dengan nama direktorinya seperti pada pemrograman Java. \textit{File} kode sumber dapat ditempatkan dimanapun pada \textit{drive}.

\subsection{Deklarasi Fungsi}

Deklarasi fungsi tanpa parameter dan tanpa nilai balikkan (\textit{return}) akan terlihat seperti contoh kode berikut :

\begin{lstlisting}
fun cetak(): Unit {
	println("Hai, apa kabar")
}
\end{lstlisting}

Atau deklarasi \texttt{Unit} dapat dihilangkan dengan kode akan terlihat seperti ini :

\begin{lstlisting}
fun cetak() {
	println("Hai, apa kabar")
}
\end{lstlisting}

Untuk deklarasi fungsi dengan parameter akan terlihat seperti contoh kode berikut :

\begin{lstlisting}
fun tambah(a: Int, b: Int): Int {
	return a + b
}
\end{lstlisting}

Fungsi yang sama seperti diatas dapat dibuat lebih ringkas dengan nilai balikan \textit{return} yang sudah diprediksi oleh Kotlin, kodenya menjadi seperti berikut ini :

\begin{lstlisting}
fun tambah(a: Int, b: Int) = a + b
\end{lstlisting}

Untuk pembahasan lebih lanjut mengenai fungsi, akan dijabarkan dalam bagian tersendiri dalam buku ini.

\subsection{Deklarasi Variabel}

Deklarasi variabel dapat dilakukan untuk 2 (dua) cara. Yang pertama adalah variabel yang hanya dapat diisi satu kali, dan ada yang dapat diisi berkali-kali. 

Kode untuk deklarasi variabel yang hanya dapat diisi 1 (satu) kali adalah sebagai berikut :

\begin{lstlisting}
val a: Int = 2 
// atau 
val c = 2
// atau
val d: Int
d = 5
\end{lstlisting}

Untuk deklarasi variabel yang dapat diubah, kodenya adalah sebagai berikut :

\begin{lstlisting}
var e = 2
e *= 2
\end{lstlisting}

\subsection{Deklarasi Komentar}

Seperti bahasa pemrograman Java dan Javascript, Kotlin juga menyediakan komentar dalam bentuk komentar baris dan komentar multi-baris. Kode untuk komentar satu baris adalah sebagai berikut :

\begin{lstlisting}
// ini komentar 1 baris
\end{lstlisting}

Untuk kode komentar multi-bari adalah sebagai berikut :

\begin{lstlisting}
/*  ini komentar
    multi baris */
\end{lstlisting}

Namun tidak seperti bahasa pemrograman Java, komentar di Kotlin dapat bersarang bertingkat.

\section{Logat}

Beberapa logat yang biasa digunakan di Kotlin adalah seperti di bawah ini.

\subsection{Membuat Kelas Data}

Kelas data ini biasa digunakan untuk pembuatan kelas \textit{entity}. Contoh kodenya adalah sebagai berikut :

\begin{lstlisting}
data class Pegawai(val nim: String, val nama: String)
\end{lstlisting}

Dengan menambahkan deklarasi \texttt{data} di depan kelas, maka untuk kelas Pegawai ini akan disediakan fungsi-fungsi berikut secara otomatis :

\begin{itemize}
\item \textit{Getters} dan \textit{Setter} untuk seluruh properti
\item \textit{Method} \texttt{equals}.
\item \textit{Method} \texttt{hashCode}
\item \textit{Method} \texttt{toString}
\item \textit{Method} \texttt{copy}
\end{itemize}

\subsection{Nilai \textit{Default} Untuk Parameter Fungsi}

Pada saat deklarasi fungsi, sebetulnya parameter dapat kita isikan dengan nilai \textit{default} seperti berikut :

\begin{lstlisting}
fun isiData(nama: String, kelamin: Int = 0) {
	...
}
\end{lstlisting}

Nantinya parameter \texttt{kelamin} akan terisi otomatis dengan \texttt{0}

Contoh kodenya adalah sebagai berikut :

\begin{lstlisting}
fun main(args: Array<String>) {
	val nama = "tamami"
	println("Halo, $nama")
	
	isiData(nama)
}

fun isiData(nama: String, kelamin: Int = 0) {
	println(kelamin)
}
\end{lstlisting}

Hasil keluarannya adalah sebagai berikut :

\begin{lstlisting}
Halo, tamami
0
\end{lstlisting}

Penjelasannya adalah sebagai berikut, pada baris pertama menghasilkan keluaran teks \texttt{Halo, tamami}, yang sebetulnya hasil dari eksekusi perintah kode pada baris ke-3, yaitu :

\begin{lstlisting}
println("Halo, $nama")
\end{lstlisting}

Dimana pemanggilan variabel \texttt{\$nama} pada baris ke-2 dari \textit{source code} terjadi, dan yang ditampilkan di layar monitor adalah isi dari variabel \texttt{\$nama}, yaitu \texttt{tamami}.

Sedangkan pada baris kedua dari hasil keluaran, yaitu \texttt{0}, adalah hasil dari eksekusi kode pada bari ke-9, di dalam fungsi \texttt{isiData}, tepatnya pada perintah berikut :

\begin{lstlisting}
println(kelamin)
\end{lstlisting}

Kenapa hasil keluarannya adalah \texttt{0}, alurnya adalah seperti ini, pada saat pemanggilan fungsi \texttt{isiData(nama)} pada baris ke-5, parameter \texttt{nama} pada fungsi \texttt{isiData} ini terisi dengan nilai \texttt{tamami}, karena parameter kedua, yaitu \texttt{kelamin} tidak disertakan pada pemanggilannya pada baris ke-5, sehingga parameter \texttt{kelamin} akan terisi otomatis dengan nilai \texttt{0} sebagaimana deklarasinya pada baris ke-8.

\subsection{\textit{Interpolasi} Teks}

\textit{Interpolasi} atau penyisipan teks akan terlihat seperti baris perintah berikut ini :

\begin{lstlisting}
val nama: String = "tamami"
println("name $nama")
\end{lstlisting}

Nantinya sisipan teks dengan kode \texttt{\$nama} akan terisi oleh variabel \texttt{nama} yang telah dideklarasikan sebelumnya.

\subsection{Pemeriksaan Instan}

Pada bahasa Kotlin, kita dapat melakukan pemeriksaan tipe data secara instan, formatnya adalah seperti kode berikut :

\begin{lstlisting}
when(x) {
	is String -> ...
	is Int -> ...
	is KelasSaya -> ...
	else -> ...
}
\end{lstlisting}

Artinya nanti isi dari variabel \texttt{x} akan dipilah, apakah merupakan tipe data \texttt{String}, \texttt{Int}, merupakan instan dari kelas \texttt{KelasSaya}, atau berupa tipe data atau kelas lain.

Contoh nyata dari penggunaan kode di atas adalah sebagai berikut :

\begin{lstlisting}
fun main(args: Array<String>) {
	val x: Any = 2
	
	when (x) {
		is String -> println("Jawaban String")
		is Int -> println("Jawaban Int")
		else -> println("lainnya")
	}
}
\end{lstlisting}

Pada kode di atas, tipe data dari variabel \texttt{x} adalah \texttt{Any}, yang artinya bisa berupa tipe data apapun, atau instan dari kelas apapun. Lalu diisikan nilai awal berupa angka 2 (dua).

Selanjutnya kode akan melakukan seleksi tipe data \texttt{x} pada baris ke-4 dengan perintah \texttt{when (x)}, kemudian melakukan pemeriksaan, apabila tipe data dari variabel \texttt{x} adalah \texttt{String} maka akan dicetak seperti pada baris ke-5 dari kode di atas, tapi ternyata memang tipe data yang tepat adalah pada baris ke-6 sehingga program yang kita bangun akan mencetak \texttt{Jawaban Int} di layar, karena variabel \texttt{x} berisi angka 2 (dua).

\subsection{Penggunaan \textit{Range}}

Penggunaan \textit{range} biasanya untuk melakukan iterasi atau perulangan, beberapa contohnya akan dikerjakan dengan kode berikut :

\begin{lstlisting}
for(i in 1..10) { println("data ke-$i = $i") }
\end{lstlisting}

Kode tersebut akan menghasilkan keluaran di monitor dimana nilai \texttt{i} dari 1 sampai dengan 10 sebagai berikut :

\begin{lstlisting}
data ke-1 = 1
data ke-2 = 2
data ke-3 = 3
data ke-4 = 4
data ke-5 = 5
data ke-6 = 6
data ke-7 = 7
data ke-8 = 8
data ke-9 = 9
data ke-10 = 10
\end{lstlisting}

Contoh penggunaan \textit{range} yang lain adalah seperti kode berikut :

\begin{lstlisting}
for(i in 1 until 10) { println("data ke-$i = $i") }
\end{lstlisting}

Sama seperti kode sebelumnya, hanya saja kali ini nilai \texttt{i} adalah antara 1 sampai dengan 9, angka 10 tidak masuk dalam kualifikasi proses cetak ke monitor. Berikut adalah hasil keluarannya di monitor :

\begin{lstlisting}
data ke-1 = 1
data ke-2 = 2
data ke-3 = 3
data ke-4 = 4
data ke-5 = 5
data ke-6 = 6
data ke-7 = 7
data ke-8 = 8
data ke-9 = 9
\end{lstlisting}

Contoh penggunaan \textit{range} dengan beberapa pola lompatan atau kelipatan angka adalah sebagai berikut :

\begin{lstlisting}
for(i in 1..10 step 3) { println("data ke-$i") }
\end{lstlisting}

Maksudnya adalah mencetak deretan angka yang dimulai dari 1 dengan kelipatan 3 sampai nilai \texttt{i} sama dengan 10. Berikut adalah hasil keluaran dari kode tersebut :

\begin{lstlisting}
data ke-1
data ke-4
data ke-7
data ke-10
\end{lstlisting}

Contoh penggunaan \textit{range} untuk perulangan yang mundur ke nilai yang lebih kecil adalah sebagai berikut :

\begin{lstlisting}
for(i in 10 downTo 1) { println("data ke-$i") }
\end{lstlisting}

Penjelasan dari kode tersebut dijelaskan dengan hasil keluaran di layar monitor sebagai berikut :

\begin{lstlisting}
data ke-10
data ke-9
data ke-8
data ke-7
data ke-6
data ke-5
data ke-4
data ke-3
data ke-2
data ke-1
\end{lstlisting}

\subsection{\textit{Read-only List}}

Artinya adalah membuat \textit{list} yang tidak dapat diubah isinya, contoh deklarasinya adalah sebagai berikut :

\begin{lstlisting}
val list = listOf("data A", "data B", "data C")
\end{lstlisting}

Untuk lebih jelasnya, kita akan melihat kode contoh sebagai berikut :

\begin{lstlisting}
fun main(args: Array<String>) {
	val list = listOf("A", "B", "C");
	
	for(i in list) {
		println("data $i")	
	]
}
\end{lstlisting}

Kode tersebut, pada baris ke-2 adalah menyiapkan objek \textit{list} dan diisikan langsung dengan data menggunakan perintah \texttt{listOf}, selanjutnya seluruh data dicetak ke layar monitor sebagaimana tampilan berikut :

\begin{lstlisting}
data A
data B
data C
\end{lstlisting}

\subsection{\textit{Read-only Map}}
\subsection{Mengakses \textit{Map}}
\subsection{Jalan Pintas Perintah \texttt{if not null}}
\subsection{Jalan Pintas Perintah \texttt{if not null and else}}
\subsection{Eksekusi Perintah \texttt{if null}}
\subsection{Eksekusi Perintah \texttt{if not null}}
\subsection{Kembalikan Pada Perintah \texttt{when}}
\subsection{Ekspresi \texttt{try catch}}
\subsection{Ekspresi \texttt{if}}

\section{Adat}
\chapter{Dasar-Dasar}

\section{Tipe Data}

Sebagaimana kebanyakan bahasa pemrograman, mengenali tipe data adalah hal yang penting untuk diketaui, karena perbedaan tipe data dapat menyebabkan perbedaan operasi yang dapat dilakukan kepadanya.

Tipe data secara garis besar dapat dikelompokan menjadi : angka, karakter, \textit{boolean}, dan larik.

\subsection{Angka}

\subsection{Karakter}

\subsection{\textit{Boolean}}

\subsection{Larik}


\section{Paket}

\section{Mengatur Alur}


\chapter{Kelas dan Objek}

\section{Kelas}

Kelas di Kotlin dideklarasikan dengan kata kunci \texttt{class}.

\section{Properti}

\section{\textit{Interface}}

\section{\textit{Visibility Modifiers}}

\section{Ekstensi}

\section{Kelas Data}

\section{Kelas Tertutup}

\section{Generik}

\section{Kelas Bersarang}

\section{Kelas \textit{Enum}}

\section{Ekspresi Objek dan Deklarasi}

\section{Delegasi}

\section{Mendelegasikan Properti}
\chapter{Fungsi dan Lamda}

\section{Fungsi}

\section{Fungsi Lanjutan dan Lamda}

\section{Fungsi Sebaris}

\section{\textit{Coroutines}}
\chapter{Java Interoperabilitas}

Karna Kotlin melakukan kompilasi ke dalam kelas Java, maka sebetulnya Kotlin mampu untuk menggunakan pustaka-pustaka yang ditulis dan dibangun menggunakan bahasa Java. Begitu pula sebaliknya.

\section{Gunakan Java di Kotlin}

Kotlin dibangun dengan memikirkan penggabungannya dengan pustaka Java. Kode yang dibangun di Java dapat dengan mudah digunakan di Kotlin, begitu pula sebaliknya. Coba perhatikan kode yang ditulis dalam bahasa Kotlin yang menggunakan pustaka \texttt{ArrayList} dari Java :

\begin{lstlisting}
import java.util.ArrayList;

fun main(args: Array<String>) {
	val list = ArrayList<Int>()
	
	list.add(1)
	list.add(4)
	list.add(7)
	
	for(i in list) {
		println(i)
	}
}
\end{lstlisting}

Hasil keluaran dari kode di atas adalah sebagai berikut :

\begin{lstlisting}
1
4
7
\end{lstlisting}

Terlihat bahwa kode di atas menggunakan kelas \texttt{ArrayList} yang ada pada pustaka Java.

\subsection{Fungsi \textit{Getter} dan \textit{Setter}}

Untuk fungsi \textit{getter} dan \textit{setter}, di Kotlin akan dikenal sebagai properti. Jadi misalkan ada fungsi \textit{getter} dan \textit{setter}-nya di Java, cukup dipanggil nama propertinya saja.

Perhatikan kode Java berikut ini :

\begin{lstlisting}
public class Pegawai {
	private String nama;
	
	public void setNama(String nama) {
		this.nama = nama;
	}
	
	public String getNama() {
		return nama;
	}
}
\end{lstlisting}

Kode tersebut dapat di\textit{compile} dengan \texttt{javac} kemudian nantinya akan kita gunakan pada kode Kotlin berikut :

\begin{lstlisting}
fun main(args: Array<String>) {
	val pegawai = Pegawai()
	
	pegawai.nama = "tamami"
	
	println(pegawai.nama)
}
\end{lstlisting}

Yang perlu di perhatikan untuk \textit{compile} kode di atas adalah, kelas \texttt{Pegawai} yang dibuat dengan kode Java harus dalam 1 (satu) direktori dengan \textit{file} Kotlin yang kita buat.

\textit{Compile} terlebih dahulu kelas \texttt{Pegawai} agar dapat digunakan. Untuk melakukan \textit{compile file} Kotlin harus menggunakan opsi \textit{classpath} seperti berikut ini (misalkan nama \textit{file} Kotlin yang saya buat adalah \texttt{Test.kt}) :

\begin{lstlisting}
kotlinc -cp . Test.kt
\end{lstlisting}

Begitu pula pada saat kita akan menjalankan aplikasi Kotlin yang telah kita buat, gunakan opsi \textit{classpath} seperti berikut ini :

\begin{lstlisting}
kotlin -cp . TestKt
\end{lstlisting}

Hasil keluarannya akan tampak seperti berikut ini :

\begin{lstlisting}
tamami
\end{lstlisting}

Terlihat bahwa pada kode Kotlin yang telah kita buat, kita menggunakan kelas \texttt{Pegawai} dari bahasa Java. Kemudian melakukan akses ke properti \texttt{nama} milik kelas \texttt{Pegawai} dengan memberinya nilai seperti pada baris ke-4.

Pada saat melakukan pengambilan data seperti pada baris ke-6 pun, cukup melakukan akses ke nama propertinya yang Kotlin akan menerjemahkan untuk melakukan akses atau mengambil nilai dengan fungsi \texttt{getNama} yang ada di Java.

Bagaimana jika properti di Java hanya memiliki sebuah \textit{method} \texttt{set} saja di dalamnya, maka Kotlin tidak akan bisa melakukan akses terhadap properti ini, yang dapat kita lakukan adalah dengan melakukan akses langsung terhadap nama fungsinya. Dijelaskan dengan kode (kita ubah kode Java sebelumnya) adalah sebagai berikut :

\begin{lstlisting}
public class Pegawai {
	private String nama;
	
	public void setNama(String nama) {
		this.nama = nama;
	}
}
\end{lstlisting}

Maka kode di Kotlin akan melakukan akses dengan cara berikut :

\begin{lstlisting}
fun main(args: Array<String>) {
	val pegawai : Pegawai()
	
	pegawai.setNama("tamami")
}
\end{lstlisting}

\subsection{Nilai Kembalian \texttt{void}}

Saat sebuah \textit{method} mengembalikan nilai \texttt{void} di Java, maka dalam Kotlin akan digantikan dengan kelas \texttt{Unit} pada saat \textit{compile}.

\subsection{Kata Kunci di Kotlin Jadi Nama \textit{Method} di Java}

Bila di Java menggunakan nama \textit{method} yang sama dengan kata kunci di Kotlin seperti \texttt{is}, \texttt{in}, \texttt{object}, dan lainnya, kita masih dapat menggunakannya dengan cara memberikan tanda \textit{backtick}. Berikut contoh kode di Javanya :

\begin{lstlisting}
public class Pegawai {
	private String bagian;
	
	public Pegawai(String bagian) {
		this.bagian = bagian;
	}
		
	public boolean in(String bagian) {
		if(bagian.equals(this.begin)) return true;
		else return false;
	}
}
\end{lstlisting}

Kita lihat pada baris ke-8 bahwa \textit{method} dengan nama \texttt{in} ada di Java, untuk melakukan akses terhadap \textit{method} ini di Kotlin adalah dengan cara berikut ini :

\begin{lstlisting}
fun main(args: Array<String>) {
	val pegawai = Pegawai("dattap")
	
	if(pegawai.`in`("dattap")) println("betul") else println("lain")
}
\end{lstlisting}

Untuk memanggil \textit{method} \texttt{in} seperti terlihat pada baris ke-4 di atas.

\subsection{\textit{Null-Safety}}

Di Java ada kemungkinan variabel atau properti yang ada dalam sebuah kelas bernilai \texttt{null} sedangkan Kotlin akan menjaga agar tidak ada satu variabel atau properti yang bernilai \texttt{null} pada saat melakukan \textit{compile}, lalu bagaimana solusinya, perhatikan 2 (dua) kode berikut.

Kode yang pertama di tulis dalam Java dan memiliki peluang untuk memberikan nilai \texttt{null} pada kelas yang melakukan akses terhadap properti \texttt{nama}. Berikut kodenya :

\begin{lstlisting}
public class Pegawai {
	private String nama;
	
	public void setNama(String nama) {
		this.nama = nama;
	}
	
	public String getNama() {
		return nama;
	}
}
\end{lstlisting}

Kita dapat menggunakan operator tanda tanya (\texttt{?}) untuk menampung data yang mungkin akan menghasilkan nilai \texttt{null} dari kelas \texttt{Pegawai} di atas, berikut kodenya di Kotlin :

\begin{lstlisting}
import java.util.ArrayList

fun main(args: Array<String>) {
	val list = ArrayList<String>()
	list.add("data")
	
	val pegawai = Pegawai()
	
	val boleh: String? = pegawai.nama
	val tidak: String = list[0]
	
	println(boleh)
	println(tidak)
}
\end{lstlisting}

Hasil keluaran dari kode program di atas adalah sebagai berikut :

\begin{lstlisting}
null
data
\end{lstlisting}

Pada baris ke-9 dari kode Kotlin yang kita buat, ada operator tanda tanya (\texttt{?}) disana yang akan memberikan kelonggaran, atau melewatkan pemeriksaan atas properti \texttt{nama} milik kelas \texttt{Pegawai} yang dimungkinkan bernilai \texttt{null}. Dan benar saja, pada saat dicetak seperti pada baris ke-12, hasilnya adalah \texttt{null}.

Selain itu, ada saatnya sebuah tipe data yang tidak dapat disebutkan secara eksplisit, apakah \texttt{null} atau apakah nilainya ada, yang biasa disebut tipe \textit{platform}. Untuk kasus ini bisa menggunakan notasi tanda seru (\texttt{!}) setelah tipe datanya. Misalkan untuk tipe data \texttt{T}, dapat menggunakan \texttt{T!} yang artinya dapat berupa data yang mungkin \texttt{null} seperti notasi \texttt{T?} atau tipe data yang pasti bukan \texttt{null} seperti tanda \texttt{T}.

\subsection{Persamaan Tipe Data}

Dari percobaan sebelumnya, kita telah ketahui bahwa tipe data primitif Java tidak diterjemahkan seperti data primitif di Kotlin, namun akan dipetakan ke bentuk kelas, beberapa pemetaan yang dilakukan untuk tipe data primitif adalah sebagai berikut :

\begin{center}
\begin{tabular}{|l|l|}
\hline
\textbf{Java} & \textbf{Kotlin} \\
\hline
byte & kotlin.Byte \\
\hline
short & kotlin.Short \\
\hline
int & kotlin.Int \\
\hline
long & kotlin.Long \\
\hline
char & kotlin.Char \\
\hline 
float & kotlin.Float \\
\hline
double & kotlin.Double \\
\hline
boolean & kotlin.Boolean \\
\hline
\end{tabular}
\end{center}

Beberapa tipe data yang bukan termasuk ke tipe data primitif pun akan dipetakan sebagai berikut di Kotlin :

\begin{center}
\begin{tabular}{|l|l|}
\hline
\textbf{Java} & \textbf{Kotlin} \\
\hline
java.lang.Object & kotlin.Any! \\
\hline
java.lang.Cloneable & kotlin.Cloneable! \\
\hline
java.lang.Comparable & kotlin.Comparable! \\
\hline
java.lang.Enum & kotlin.Enum! \\
\hline
java.lang.Annotation & kotlin.Annotation! \\
\hline
java.lang.Deprecated & kotlin.Deprecated! \\
\hline
java.lang.CharSequence & kotlin.CharSequence! \\
\hline
java.lang.String & kotlin.String! \\
\hline
java.lang.Number & kotlin.Number! \\
\hline
java.lang.Throwable & kotlin.Throwable! \\
\hline
\end{tabular}
\end{center}

Beberapa tipe data primitif yang dibentuk dalam kelas juga akan dipetakan di Kotlin seperti berikut ini :

\begin{center}
\begin{tabular}{|l|l|}
\hline
\textbf{Java} & \textbf{Kotlin} \\
\hline
java.lang.Byte & kotlin.Byte? \\
\hline
java.lang.Short & kotlin.Short? \\
\hline
java.lang.Integer & kotlin.Int? \\
\hline
java.lang.Long & kotlin.Long? \\
\hline
java.lang.Char & kotlin.Char? \\
\hline
java.lang.Float & kotlin.Float? \\
\hline
java.lang.Double & kotlin.Double? \\
\hline
java.lang.Boolean & kotlin.Boolean? \\
\hline
\end{tabular}
\end{center}

\textit{Collection} di Kotlin bisa berupa \textit{read-only} atau dapat diubah, jadi ketentuan \textit{collection} di Java akan berlaku seperti ini :

\begin{center}
\begin{tabular}{|l|l|l|l|}
\hline
\textbf{Java} & \textbf{Kotlin \textit{read-only}} & \textbf{Kotlin dapat diubah} & \textbf{Kotlin \textit{Platform}} \\
\hline
Iterator<T> & Iterator<T> & MutableIterator<T> & (Mutable) Iterator<T>! \\
\hline
Iterable<T> & Iterable<T> & MutableIterable<T> & (Mutable) Iterable<T>! \\
\hline
Collection<T> & Collection<T> & MutableCollection<T> & (Mutable) Collection<T>! \\
\hline
Set<T> & Set<T> & MutableSet<T> & (Mutable) Set<T>! \\
\hline
List<T> & List<T> & MutableList<T> & (Mutable) List<T>! \\
\hline
ListIterator<T> & ListIterator<T> & MutableListIterator<T> & (Mutable) ListIterator<T>! \\
\hline
Map<K, V> & Map<K, V> & MutableMap<K, V> & (Mutable) Map<K, V>! \\
\hline
Map.Entry<K, V> & Map.Entry<K, V> & MutableMap.MutableEntry<K, V> & (Mutable) Map.(Mutable)Entry<K, V>
\end{tabular}
\end{center}

Semua kelas di atas berada dalam paket \texttt{kotlin.collections}.

Untuk larik sendiri, di Kotlin akan diterjemahkan sebagai berikut :

\begin{center}
\begin{tabular}{|l|l|}
\hline
\textbf{Java} & \textbf{Kotlin} \\
\hline
int[] & kotlin.IntArray! \\
\hline
String[] & kotlin.Array<(out) String>! \\
\hline
\end{tabular}
\end{center}

\subsection{Java Generik}

Generik di Kotlin seperti pembahasan pada bagian \ref{secGenerik}, berbeda dengan kondisi generik di Java. Beberapa konversi yang dilakukan antara generik di Java dengan Kotlin adalah sebagai berikut :

\begin{itemize}
	\item \textit{Wildcard} di Java akan diterjemahkan berikut :
		\begin{itemize}
			\item \texttt{MyMethod<? extends Kelas>} menjadi \texttt{MyMethod<out Kelas!>!}.
			\item \texttt{MyMethod<? super Kelas>} menjadi \texttt{MyMethod<in Kelas!>!}
		\end{itemize}	
	\item Untuk tipe data mentah, yang biasanya tidak dideklarasikan akan menjadi seperti berikut :
		\begin{itemize}
			\item \texttt{ArrayList} menjadi \texttt{ArrayList<*>!}
		\end{itemize}
\end{itemize}

\subsection{Larik Java}

Untuk larik, ingatlah bahwa larik di Kotlin berbeda dengan larik di Java. Di Kotlin tidak diperbolehkan mengisikan data dari sub-kelas atau super-kelas, contohnya misalkan deklarasi yang disebutkan adalah \texttt{Array<Any>}, deklarasi tersebut tidak bisa diisikan dengan data seperti \texttt{Array<String>}. Dan untuk tipe data primitif di Java, ada kelas yang bertugas menangani masing-masing tipe data tersebut dalam larik seperti \texttt{IntArray}, \texttt{CharArray}, \texttt{FloatArray} dan seterusnya.

Gambaran kodenya akan terlihat seperti percobaan berikut, pertama kita buat dahulu kode dari Java dimana salah satu \textit{method} membutuhkan larik primitif \texttt{int} untuk di cetak ke layar monitor. Kode Javanya sebagai berikut :

\begin{lstlisting}
public class Pencetak {
	public void go(int[] data) {
		for(int i=0; i<data.length; i++) {
			System.out.println(data[i]);
		}
	}
}
\end{lstlisting}

Kode Kotlin yang menggunakan kelas \texttt{Pencetak} tersebut adalah sebagai berikut :

\begin{lstlisting}
fun main(args: Array<String>) {
	val data = intArrayOf(0,2,4,6)
	val pencetak = Pencetak()
	
	pencetak.go(data)
}
\end{lstlisting}

Hasil yang dikeluarkan oleh kode program di atas adalah sebagai berikut : 

\begin{lstlisting}
0
2
4
6
\end{lstlisting}

\subsection{Varargs di Java}

\textit{Varargs} di Java sebetulnya mirip dengan larik, hanya saja deklarasinya memiliki perbedaan dan lebih terlihat dinamis. Contoh berikut akan memberikan gambaran bahwa \textit{varargs} di Java dapat diakses dengan cara yang mirip dengan larik di Java apabila digunakan di dalam kode Kotlin. Berikut adalah kode dengan bahasa pemrograman Java yang memiliki parameter \textit{varargs} :

\begin{lstlisting}
public class Pencetak {
	public void go(int... data) {
		for(int i=0; i<data.length; i++) {
			System.out.println(data[i]);
		}
	}
}
\end{lstlisting}

Sedangkan kode di Kotlin untuk melakukan akses terhadap fungsi \texttt{go} dengan parameter \texttt{data} yang berupa \textit{varargs} adalah sebagai berikut :

\begin{lstlisting}
fun main(args: Array<String>) {
	val data = intArrayOf(0, 2, 4, 6)
	val pencetak = Pencetak()
	pencetak.go(*data)
}
\end{lstlisting}

Hasil keluarannya sama seperti kode sebelumnya, perbedaan yang dapat kita lihat ada pada baris ke-4, yaitu saat melakukan pemanggilan fungsi \texttt{go} dimana ada tanda bintang (\texttt{*}) pada parameter \texttt{data} yang diberikan.

\subsection{Pemeriksaan \textit{Exception}}

Di Kotlin, \textit{exception} tidak diperiksa terlebih dahulu, artinya, saat kita menggunakan \textit{method} di Java yang kemudian menghasilkan \textit{exception}, maka kita dapat abaikan saja, karena pada saat kompile dilakukan pun tidak ada peringatan kesalahan. 

Untuk lebih jelasnya, lihatlah kode berikut, kode pertama adalah kode Java yang menghasilkan \textit{exception}, berikut kodenya :

\begin{lstlisting}
public class Pencetak {
	public void go(int... data) throws Exception {
		if(data.length > 5) throw new Exception("Banyak Amat");
		
		for(int i=0; i<data.length; i++) {
			System.out.println(data[i]);
		}
	}
}
\end{lstlisting}

Dan kode Kotlin yang memanggil \textit{method} Java \texttt{go} tersebut akan membuat kesalahan dengan mengisikan 6 (enam) parameter seperti kode berikut :

\begin{lstlisting}
fun main(args: Array<String>) {
	val data = intArrayOf(0, 2, 4, 6, 8, 10)
	val pencetak = Pencetak()	
	
	pencetak.go(*data)
}
\end{lstlisting}

Pada saat \textit{compile} dilakukan terhadap kode Kotlin tidak akan memunculkan masalah, tetapi begitu kita jalankan aplikasi Kotlin yang telah kita buat, maka baru akan muncul \textit{exception} yang dikembalikan oleh \textit{method} \texttt{go} dari kelas \texttt{Pencetak} di Java.

\subsection{\textit{Method} Kelas Objek di Java}

Pada saat kita mendeklarasikan kelas \texttt{Object} di Java, maka pada saat digunakan di Kotlin, kelas tersebut akan dikonversi ke kelas \texttt{Any} di Kotlin.

Kelas \texttt{Any} ini berbeda dengan kelas \texttt{Object} karena hanya memiliki 3 (tiga) \textit{method} saja, yaitu \texttt{toString()}, \texttt{hashCode()}, dan \texttt{equals()}. Untuk implementasi \textit{method} lain dari kelas \texttt{Object} maka perlu didefinisikan dengan \textit{fungsi ekstensi} seperti yang telah kita bahas sebelumnya.

\subsection{Mengakses \texttt{static}}

\textit{Static Member} di Java apabila digunakan di Kotlin akan dikonversi menjadi \textit{companion object} seperti penjelasan sebelumnya. Contoh kodenya adalah seperti berikut, kita akan ubah kode sebelumnya baik kode Java maupun kode di Kotlinnya, berikut adalah kode di Java yang kita ubah :

\begin{lstlisting}
public class Pencetak {
	public static void go(int... data) {
		for(int i=0; i<data.length; i++) {
			System.out.println(data[i]);
		}
	}
}
\end{lstlisting}

Kode di Kotlin yang akan menggunakan \textit{method static} di atas adalah sebagai berikut :

\begin{lstlisting}
fun main(args: Array<String>) {
	val data = intArrayOf(0, 2, 4, 6)
	
	Pencetak.go("data")
}
\end{lstlisting}

Hasilnya masih sama seperti di atas, yaitu akan mencetak seluruh angka yang ada dalam larik \texttt{data}. Yang berbeda adalah kita tidak perlu membuat instan baru seperti sebelumnya, namun langsung dapat dipanggil nama \textit{method}nya seperti pada baris ke-4.

\subsection{Java Reflection}

\textit{Java Reflection} ini sebetulnya sebuah paket yang digunakan untuk memeriksa kelas pada saat kondisi \textit{runtime}. Beberapa hal yang dapat kita lihat dari \textit{Java Reflection} ini adalah sebagai berikut :

\begin{itemize}
	\item Nama Kelas
	\item Lingkup kelas
	\item Informasi Paket
	\item Super-Kelas
	\item Implementasi \textit{Interface}
	\item Konstruktor
	\item \textit{Method}
	\item Properti / Atribut
	\item Anotasi
\end{itemize}

Untuk mudahnya, perhatikan kelas berikut yang menggunakan \textit{Java Reflection} :

\begin{lstlisting}
public class Test {
	public static void main(String args[]) {
		Class cl = Test.class;
		
		System.out.println(cl.getName());
	}
}
\end{lstlisting}

Hasil dari kode di atas adalah sebagai berikut :

\begin{lstlisting}
Test
\end{lstlisting}

Kode tersebut adalah kode Java yang menggunakan \textit{Java Reflection} seperti pada deklarasi di baris ke-3, yaitu penggunaan kelas \texttt{Class}, yang kemudian informasi dapat diakses dengan \textit{method} milik \texttt{Class}.

Hal tersebut dapat kita lakukan dalam Kotlin dengan kode sebagai berikut :

\begin{lstlisting}
fun main(args: Array<String>) {
	val data = Test()
	val nama = data::class.java
	
	println(nama.name)
}
\end{lstlisting}

Hasil dari kode di atas akan sama seperti kode Java sebelumnya, seperti berikut ini :

\begin{lstlisting}
Test
\end{lstlisting}

Penggunaan \textit{Java Reflection} ini seperti terlihat di baris ke-3, dimana pada baris ke-2 membentuk instan dari kelas \texttt{Test} milik Java, kemudian pada baris ke-3 dilakukan inspeksi terhadap kelas \texttt{Test} melalui instannya, yaitu \texttt{data}, kemudian dilakukan pencetakan nama kelas seperti pada baris ke-5.

\section{Gunakan Kotlin di Java}

Karena sebetulnya tiap proses \textit{compile} yang terjadi di Kotlin adalah merubah kode Kotlin menjadi kode biner Java, jadi sangat mudah menggunakan kode yang dibangun dengan Kotlin untuk digunakan di Java.

\subsection{Properti}

Properti di Kotlin akan diubah atau diurai menjadi elemen-elemen Java seperti berikut ini :

\begin{itemize}
	\item \textit{Method} get, dimana \textit{method} ini adalah gabungan dari nama properti yang diawali dengan kata kunci \texttt{get}.
	\item \textit{Method} set, dimana \textit{method} ini adalah gabungan dari nama properti yang diawali dengan kata kunci \texttt{set}. (Hanya berlaku untuk properti dengan kata kunci \texttt{var}).
	\item Properti dengan nama yang sama dengan tambahan kata kunci \texttt{private}.
\end{itemize}

Contohnya adalah seperti kode berikut ini :

\begin{lstlisting}
class Pegawai {
	var nama: String
}
\end{lstlisting}

Dari kelas \texttt{Pegawai} di atas, memiliki sebuah properti yang apabila di-\textit{compile} akan menjadi kode Java berikut :

\begin{lstlisting}
public class Pegawai {
	private String nama;
	
	public void setNama(String nama) {
		this.nama = nama;
	}
	
	public String getNama() {
		return nama;
	}
}
\end{lstlisting}

Secara sederhana akan terbentuk seperti kode di atas. Apabila sebuah properti diberi nama dengan awalan \texttt{is} seperti misalkan \texttt{isStudent}, maka konversinya adalah \textit{method get}-nya akan sama persis seperti nama propertinya, sedangkan \textit{method set}-nya akan merubah awalan \texttt{is} menjadi awalan \texttt{set}, sehingga nama \textit{method} akan menjadi \texttt{setStudent}. 

Aturan tersebut berlaku untuk seluruh tipe data selama nama propertinya mengikuti pola seperti itu.

\subsection{Fungsi Pada Paket}

Fungsi yang berada di dalam paket langsung, bukan di dalam kelas, akan diperlakukan sebagai fungsi \texttt{static} di Java. Termasuk di dalamnya adalah fungsi ekstensi. Berikut adalah kode Kotlin yang dibangun yang memiliki \textit{method} di bawah paket langsung:

\begin{lstlisting}
package data

fun keterangan() {
	println("Test Keterangan")
}
\end{lstlisting}

Misalkan \textit{file} untuk kode di atas diberi nama \texttt{Pegawai.kt}. Untuk kode Java yang melakukan / menggunakan akses fungsi \texttt{keterangan} dari kode Kotlin di atas adalah sebagai berikut :

\begin{lstlisting}
public class Test {
	public static void main(String args[]) {
		data.PegawaiKt.keterangan();
	}
}
\end{lstlisting}

Hasil dari \textit{compile} dan menjalankan kode di atas, keluarannya akan terlihat seperti ini :

\begin{lstlisting}
Test Keterangan
\end{lstlisting}

Hal ini mungkin dicapai karena memang kode Kotlin yang dibuat bertugas hanya mencetak kata tersebut. Perhatikan baris ke-3 dari kode Java yang dibuat, ada beberapa bagian disana, dimana \texttt{data} adalah nama paketnya, \texttt{PegawaiKt} adalah hasil \textit{compile} dari \textit{file} \texttt{Pegawai.kt} karena ada fungsi di dalam paketnya. Dan yang terakhir adalah \texttt{keterangan} yang sebetulnya nama fungsi di level atau tingkatan paket dengan nama \texttt{data}.

Untuk bagian \texttt{PegawaiKt} sebetulnya dapat kita ganti agar terlihat lebih pantas dengan kata kunci \texttt{@file:JvmName()}. Berikut contoh kode programnya dalam Kotlin :

\begin{lstlisting}
@file:JvmName("Pegawai")

package data

fun keterangan() {
	println("Test Keterangan")
}
\end{lstlisting}

Sehingga kode di Java akan berubah menjadi seperti ini :

\begin{lstlisting}
public class Test {
	public static void main(String args[]) {
		data.Pegawai.keterangan();
	}
}
\end{lstlisting}

Dengan hasil yang sama, namun kata \texttt{Pegawai} tidak lagi menggunakan seperti nama sebelumnya, yaitu \texttt{PegawaiKt}.

\subsection{\textit{Field} Instan}

Jika ingin membuat sebuah properti di Kotlin persis seperti sebuah properti di Java, gunakan kata kunci \texttt{@JvmField}, batasannya adalah properti ini bukan tipe \texttt{private}, bukan pula \texttt{open}, \texttt{override} atau \texttt{const}. Jenis properti lain adalah properti \texttt{late-initialized} akan juga menjadi \textit{field} murni di Java.

\subsection{\textit{Field} Statis}

Properti yang dideklarasikan sejajar dengan nama objek atau dalam \textit{companion} objek akan didefinisikan sebagai \textit{field} statis di Java. Biasanya akan bersifat \texttt{private} juga, namun dapat dibuka dengan menggunakan kata kunci berikut :

\begin{itemize}
	\item Anotasi \texttt{@JvmField}
	\item Kata Kunci \texttt{lateinit}
	\item Kata Kunci \texttt{const}
\end{itemize}

\subsection{\textit{Method} Statis}

Seperti penjelasan sebelumnya bahwa \textit{method} yang terbentuk di bawah paket langsung akan menjadi \textit{method} statis di Java. Kotlin juga akan membentuk \textit{method} statis apabila fungsi tersebut terbentuk didalam \texttt{object} atau \textit{companion} objek jika diberikan kata kunci \texttt{@JvmStatic}.

\subsection{Lingkup}

Lingkup di Kotlin akan diterjemahkan dalam kelas Java sebagai berikut :

\begin{itemize}
	\item \texttt{private} di kelas Kotlin akan tetap \texttt{private} di kelas Java.
	\item \texttt{private} di level paket akan menjadi milik paket lokal.
	\item \texttt{protected} akan tetap sebagai \texttt{protected}, namun yang perlu diingat adalah bahwa tiap properti dengan lingkup \texttt{protected} di Java masih dapat diakses selama dalam 1 (satu) paket, namun di Kotlin tidak memperbolehkan hal ini.
	\item \texttt{internal} akan menjadi \texttt{public} di Java.
	\item \texttt{public} akan tetap menjadi \texttt{public} di Java.
\end{itemize}

\subsection{KClass}

Adakalanya kita memerlukan sebuah \textit{method} dengan parameter berupa tipe data \texttt{KClass}, tipe data ini tidak ada padanannya di Java, maka kita harus melakukan konversi secara manual seperti kelas \texttt{MainView} berikut :

\begin{lstlisting}
kotlin.jvm.JvmClassMappingKt.getKotlinClass(MainView.class)
\end{lstlisting}

\subsection{Tangani Kesamaan Ciri dengan \texttt{@JvmName}}

Apabila ada 2 (dua) fungsi yang dibuat di Kotlin seperti berikut ini :

\begin{lstlisting}
fun List<String>.filterValid(): List<String>
fun List<Int>.filterValid(): List<Int>
\end{lstlisting}

Kedua fungsi tersebut akan menjadi bentrok apabila di konversi ke Java karena Java hanya akan melihat sebuah fungsi berikut :

\begin{lstlisting}
List<String> filterValid() {}
\end{lstlisting}

Hal ini dapat diselesaikan dengan kata kunci \texttt{@JvmName} dengan perubahan sebagai berikut di kode Kotlin :

\begin{lstlisting}
fun List<String>.filterValid(): List<String>

@JvmName("filterValidInt")
fun List<Int>.filterValid(): List<Int>
\end{lstlisting}

Karena dengan tambahan kode tersebut, di Java kodenya akan menjadi seperti ini :

\begin{lstlisting}
List<String> filterValid() {}

List<Int> filterValidInt() {}
\end{lstlisting}

\subsection{Pembentukan \textit{Overload}}

Sebenarnya, saat kita menulis \textit{method} di Kotlin dengan parameter yang sudah terisi secara \textit{default}, saat diterjemahkan ke Java maka Java hanya akan mengenal satu \textit{method} tersebut dengan jumlah parameter lengkap, misalnya kode kotlin seperti ini :

\begin{lstlisting}
fun fungsi(nama: String, kelamin: String = "pria", jabatan: String = "Staf") {}
\end{lstlisting}

Maka di Java hanya akan dikenal satu fungsi dengan pola sebagai berikut :

\begin{lstlisting}
void fungsi(String nama, String kelamin, String jabatan) {}
\end{lstlisting}

Padahal seharusnya di Kotlin fungsi \texttt{fungsi} masih dapat dipanggil dengan hanya sebuah parameter saja, yaitu \texttt{nama} seperti ini :

\begin{lstlisting}
fungsi("tamami")
\end{lstlisting}

Agar kode di Java dapat dengan mudah mengikuti pola ini, maka fungsi tersebut harus diberikan kata kunci \texttt{@JvmOverloads} seperti berikut :

\begin{lstlisting}
@JvmOverloads fun fungsi(nama: String, kelamin: String = "pria", 
	jabatan: String = "Staf") {}
\end{lstlisting}

Nantinya di Java akan dapat menggunakan salah satu dari \textit{method} berikut yang terbentuk secara otomatis : 

\begin{lstlisting}
void fungsi(String nama) {}
void fungsi(String nama, String kelamin) {}
void fungsi(String nama, String kelamin, String jabatan) {}
\end{lstlisting}

\subsection{Pemeriksaan \textit{Exception}}

\subsection{\textit{Null-safety}}

\subsection{Generik}
\chapter{Perkakas}

Untuk membangun sebuah aplikasi, dewasa ini dikenal beberapa perkakas yang mempermudah dan mempercepat kita dalam mengumpulkan beberapa pustaka dalam sebuah \textit{project}.

Beberapa perkakas untuk kebutuhan ini beberapa yang terkenal diantaranya Gradle dan Maven, kita akan bahas keduanya, namun untuk penggunaan kedepan kembali lagi kepada pilihan masing-masing.

\section{Menggunakan Gradle}

Gradle ini adalah perangkat otomasi untuk membangun sebuah aplikasi yang dibangun berdasarkan DSL (\textit{Domain-Specific Language) dari bahasa Groovy. 

Gradle ini dibangun untuk dapat menangani multi-\textit{project} dengan cerdas, Gradle tahu bagian mana yang sudah terbarukan dan melakukan eksekusi terhadap paket tersebut tanpa harus melakukan eksekusi disemua bagian.

Karena Gradle berjalan di atas JVM, maka kita diharuskan untuk melakukan instalasi Java terlebih dahulu, dan versi Java yang dibutuhkan adalah versi 7 dan setelahnya.

Untuk proses instalasi dapat mengikuti petunjuk yang disediakan di laman Gradle, yaitu https://gradle.org/install. Untuk memastikan bahwa Gradle telah terpasang, dapat mengetikkan perintah berikut :

\begin{lstlisting}
gradle -v
\end{lstlisting}

Bila perintah tersebut berhasil menampilkan informasi versi Gradle, maka Gradle siap kita gunakan.

\textit{File} konfigurasi yang digunakan untuk sebuah \textit{project} menggunakan Gradle adalah \texttt{build.gradle}. Apabila kita ingin membangun sebuah aplikasi dengan bahasa Kotlin, maka isi minimal dari \textit{file} \texttt{build.gradle} ini adalah sebagai berikut :

\begin{lstlisting}
buildcript {
	ext.kotlin_version = '1.1.2'
	
	repositories {
		mavenCentral()
	}
	
	dependencies {
		classpath "org.jetbrains.kotlin:kotlin-gradle-plugin:$kotlin_version"
	}
}

apply plugin: 'kotlin'
apply plugin: 'application'

mainClassName = 'HelloKt'

defaultTasks 'run'

repositories {
	mavenCentral()
}

dependencies {
	compile "org.jetbrains.kotlin:kotlin-gradle-plugin:$kotlin_version"
}
\end{lstlisting}

Kode di atas baru isi dari konfigurasi Gradle, untuk kode Kotlin sendiri akan disimpan pada bagian / \textit{folder} sesuai struktur yang ditetapkan Gradle, yaitu didalam \textit{folder} \texttt{src/main/kotlin}.

\textit{File} kode Kotlin yang akan kita buat kita beri nama \texttt{Hello.kt} yang disimpan dalam \textit{folder} seperti di atas, yaitu \texttt{src/main/kotlin}. Berikut isi kode dari \textit{file} \texttt{Hello.kt} :

\begin{lstlisting}
fun main(args: Array<String>) {
	println("Halo")
}
\end{lstlisting}

Untuk menjalankan kode di atas, gunakan perintah berikut :

\begin{lstlisting}
gradle run
\end{lstlisting}

Nantinya akan ada beberapa persiapan yang dilakukan oleh Gradle, termasuk pengumpulan / pengunduhan beberapa pustaka yang dibutuhkan dalam \textit{project} kemudian melakukan kompilasi dan akhirnya mengeksekusi kode program Kotlin yang kita buat.

\section{Menggunakan Maven}

Maven pun sebetulnya sama dengan Gradle, hanya saja \textit{file} konfigurasi pembentuk unit \textit{project} menggunakan format XML. \textit{File} konfigurasi untuk membangun sebuah \textit{project} di Maven diberi nama \texttt{pom.xml}. 

Isi minimal dari \textit{file} tersebut untuk membangun sebuah \textit{project} Kotlin adalah sebagai berikut :

\begin{lstlisting}
<project xmlns="http://maven.apache.org/POM/4.0.0"
        xmlns:xsi="http://www.w3.org/2001/XMLSchema-instance"
        xsi:schemaLocation="http://maven.apache.org/POM/4.0.0
             http://maven.apache.org/xsd/maven-4.0.0.xsd">
    <modelVersion>4.0.0</modelVersion>

    <groupId>lab.aikibo</groupId>
    <artifactId>halo-kotlin</artifactId>
    <version>1.0</version>
 
    <properties>
        <kotlin.version>1.1.2</kotlin.version>
    </properties>

    <dependencies>
        <dependency>
            <groupId>org.jetbrains.kotlin</groupId>
            <artifactId>kotlin-stdlib</artifactId>
            <version>${kotlin.version}</version>
        </dependency>
    </dependencies>

    <build>
        <sourceDirectory>${project.basedir}/src/main/kotlin</sourceDirectory>
        <plugins>
            <plugin>
                <artifactId>kotlin-maven-plugin</artifactId>
                <groupId>org.jetbrains.kotlin</groupId>
                <version>${kotlin.version}</version>
                <executions>
                    <execution>
                        <id>compile</id>
                        <goals>
                            <goal>compile</goal>
                        </goals>
                    </execution>
                </executions>
            </plugin>
        </plugins>
    </build>
</project>
\end{lstlisting}

Lebih panjang memang dari format konfigurasi milik Gradle. Struktur penempatan \textit{file} kode juga sama seperti Gradle, yaitu di \textit{folder} \texttt{src/main/kotlin}. \textit{File} yang kita coba untuk Maven ini sama persis dengan Gradle, isinya sebagai berikut :

\begin{lstlisting}
fun main(args: Array<String>) {
	println("Halo")
}
\end{lstlisting}

Untuk menjalankan \textit{project} ini, perlu dilakukan \textit{compile} terlebih dahulu dengan perintah berikut :

\begin{lstlisting}
mvn compile
\end{lstlisting}

Nantinya Maven akan melakukan pengumpulan / pengunduhan pustaka, kemudian melakukan \textit{compile} terhadap kode yang kita buat. Hasil dari proses ini akan terbentuk sebuah \textit{folder} dengan nama \texttt{target} yang isinya tentu saja sudah merupakan kumpulan pustaka yang dibutuhkan dan hasil dari kompilasi \textit{file} kode yang kita susun.

Untuk menjalankan \textit{project}, gunakan perintah berikut :

\begin{lstlisting}
mvn exec:java -Dexec.mainClass="HelloKt"
\end{lstlisting}

Kenapa menggunakan Java, karena memang sama saja, setelah \textit{file} kode Kotlin kita \textit{compile} akan menghasilkan \textit{file} biner kode Java.

Selanjutnya akan kita bangun kode lengkap dari sebuah \textit{project} aplikasi \textit{chat} dengan menggunakan Kotlin. 
\chapter{Contoh Kasus \\ Aplikasi Chat}

\end{document}