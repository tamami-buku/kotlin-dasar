\chapter{Perkakas}

Untuk membangun sebuah aplikasi, dewasa ini dikenal beberapa perkakas yang mempermudah dan mempercepat kita dalam mengumpulkan beberapa pustaka dalam sebuah \textit{project}.

Beberapa perkakas untuk kebutuhan ini beberapa yang terkenal diantaranya Gradle dan Maven, kita akan bahas keduanya, namun untuk penggunaan kedepan kembali lagi kepada pilihan masing-masing.

\section{Menggunakan Gradle}

Gradle ini adalah perangkat otomasi untuk membangun sebuah aplikasi yang dibangun berdasarkan DSL (\textit{Domain-Specific Language) dari bahasa Groovy. 

Gradle ini dibangun untuk dapat menangani multi-\textit{project} dengan cerdas, Gradle tahu bagian mana yang sudah terbarukan dan melakukan eksekusi terhadap paket tersebut tanpa harus melakukan eksekusi disemua bagian.

Karena Gradle berjalan di atas JVM, maka kita diharuskan untuk melakukan instalasi Java terlebih dahulu, dan versi Java yang dibutuhkan adalah versi 7 dan setelahnya.

Untuk proses instalasi dapat mengikuti petunjuk yang disediakan di laman Gradle, yaitu https://gradle.org/install. Untuk memastikan bahwa Gradle telah terpasang, dapat mengetikkan perintah berikut :

\begin{lstlisting}
gradle -v
\end{lstlisting}

Bila perintah tersebut berhasil menampilkan informasi versi Gradle, maka Gradle siap kita gunakan.

\textit{File} konfigurasi yang digunakan untuk sebuah \textit{project} menggunakan Gradle adalah \texttt{build.gradle}. Apabila kita ingin membangun sebuah aplikasi dengan bahasa Kotlin, maka isi minimal dari \textit{file} \texttt{build.gradle} ini adalah sebagai berikut :

\begin{lstlisting}
buildcript {
	ext.kotlin_version = '1.1.2'
	
	repositories {
		mavenCentral()
	}
	
	dependencies {
		classpath "org.jetbrains.kotlin:kotlin-gradle-plugin:$kotlin_version"
	}
}

apply plugin: 'kotlin'
apply plugin: 'application'

mainClassName = 'HelloKt'

defaultTasks 'run'

repositories {
	mavenCentral()
}

dependencies {
	compile "org.jetbrains.kotlin:kotlin-gradle-plugin:$kotlin_version"
}
\end{lstlisting}

Kode di atas baru isi dari konfigurasi Gradle, untuk kode Kotlin sendiri akan disimpan pada bagian / \textit{folder} sesuai struktur yang ditetapkan Gradle, yaitu didalam \textit{folder} \texttt{src/main/kotlin}.

\textit{File} kode Kotlin yang akan kita buat kita beri nama \texttt{Hello.kt} yang disimpan dalam \textit{folder} seperti di atas, yaitu \texttt{src/main/kotlin}. Berikut isi kode dari \textit{file} \texttt{Hello.kt} :

\begin{lstlisting}
fun main(args: Array<String>) {
	println("Halo")
}
\end{lstlisting}

Untuk menjalankan kode di atas, gunakan perintah berikut :

\begin{lstlisting}
gradle run
\end{lstlisting}

Nantinya akan ada beberapa persiapan yang dilakukan oleh Gradle, termasuk pengumpulan / pengunduhan beberapa pustaka yang dibutuhkan dalam \textit{project} kemudian melakukan kompilasi dan akhirnya mengeksekusi kode program Kotlin yang kita buat.

\section{Menggunakan Maven}

Maven pun sebetulnya sama dengan Gradle, hanya saja \textit{file} konfigurasi pembentuk unit \textit{project} menggunakan format XML. \textit{File} konfigurasi untuk membangun sebuah \textit{project} di Maven diberi nama \texttt{pom.xml}. 

Isi minimal dari \textit{file} tersebut untuk membangun sebuah \textit{project} Kotlin adalah sebagai berikut :

\begin{lstlisting}
<project xmlns="http://maven.apache.org/POM/4.0.0"
        xmlns:xsi="http://www.w3.org/2001/XMLSchema-instance"
        xsi:schemaLocation="http://maven.apache.org/POM/4.0.0
             http://maven.apache.org/xsd/maven-4.0.0.xsd">
    <modelVersion>4.0.0</modelVersion>

    <groupId>lab.aikibo</groupId>
    <artifactId>halo-kotlin</artifactId>
    <version>1.0</version>
 
    <properties>
        <kotlin.version>1.1.2</kotlin.version>
    </properties>

    <dependencies>
        <dependency>
            <groupId>org.jetbrains.kotlin</groupId>
            <artifactId>kotlin-stdlib</artifactId>
            <version>${kotlin.version}</version>
        </dependency>
    </dependencies>

    <build>
        <sourceDirectory>${project.basedir}/src/main/kotlin</sourceDirectory>
        <plugins>
            <plugin>
                <artifactId>kotlin-maven-plugin</artifactId>
                <groupId>org.jetbrains.kotlin</groupId>
                <version>${kotlin.version}</version>
                <executions>
                    <execution>
                        <id>compile</id>
                        <goals>
                            <goal>compile</goal>
                        </goals>
                    </execution>
                </executions>
            </plugin>
        </plugins>
    </build>
</project>
\end{lstlisting}

Lebih panjang memang dari format konfigurasi milik Gradle. Struktur penempatan \textit{file} kode juga sama seperti Gradle, yaitu di \textit{folder} \texttt{src/main/kotlin}. \textit{File} yang kita coba untuk Maven ini sama persis dengan Gradle, isinya sebagai berikut :

\begin{lstlisting}
fun main(args: Array<String>) {
	println("Halo")
}
\end{lstlisting}

Untuk menjalankan \textit{project} ini, perlu dilakukan \textit{compile} terlebih dahulu dengan perintah berikut :

\begin{lstlisting}
mvn compile
\end{lstlisting}

Nantinya Maven akan melakukan pengumpulan / pengunduhan pustaka, kemudian melakukan \textit{compile} terhadap kode yang kita buat. Hasil dari proses ini akan terbentuk sebuah \textit{folder} dengan nama \texttt{target} yang isinya tentu saja sudah merupakan kumpulan pustaka yang dibutuhkan dan hasil dari kompilasi \textit{file} kode yang kita susun.

Untuk menjalankan \textit{project}, gunakan perintah berikut :

\begin{lstlisting}
mvn exec:java -Dexec.mainClass="HelloKt"
\end{lstlisting}

Kenapa menggunakan Java, karena memang sama saja, setelah \textit{file} kode Kotlin kita \textit{compile} akan menghasilkan \textit{file} biner kode Java.

Selanjutnya akan kita bangun kode lengkap dari sebuah \textit{project} aplikasi \textit{chat} dengan menggunakan Kotlin. 