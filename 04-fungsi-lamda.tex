\chapter{Fungsi dan Lamda}

\section{Fungsi}

Fungsi sebetulnya adalah sekumpulan kode program yang spesifik untuk menyelesaikan satu permasalahan. Fungsi dalam Kotlin ditandai dengan kata kunci \texttt{fun}.

Contoh kode berikut akan memberikan gambaran deklarasi dan penggunaan fungsi pada Kotlin :

\begin{lstlisting}
fun main(args: Array<String>) {
	cetak("tamami")
}

fun cetak(nama: String) {
	println("nama : $nama")
}
\end{lstlisting}

Hasil keluaran dari kode program di atas adalah sebagai berikut :

\begin{lstlisting}
nama : tamami
\end{lstlisting}

Bila kita perhatikan, sebetulnya deklarasi fungsi \texttt{main} pada baris ke-1 sampai dengan baris ke-3 adalah deklarasi fungsi yang selalu kita sertakan, karena fungsi ini wajib ada apabila aplikasi ingin kita jalankan.

Fungsi \texttt{main} ini hanya menerima sebuah argumen bertipe \texttt{Array} yang sebetulnya akan terisi apabila kita menyertakan parameter pada saat menjalankan aplikasi. Kita akan coba ini nanti.

Yang kedua adalah fungsi dengan nama \texttt{cetak} yang deklarasinya ada pada baris ke-5 sampai dengan baris ke-7. Fungsi ini sederhana, hanya meminta sebuah parameter bertipe \texttt{String} dan mencetak parameter tersebut ke monitor.

Sekarang coba ubah kode di atas menjadi seperti berikut :

\begin{lstlisting}
fun main(args: Array<String>) {
	cetak(args[0])
}

fun cetak(nama: String) {
	println("nama : $nama")
}
\end{lstlisting}

Setelah melakukan \textit{compile}, jalankan aplikasi dengan memberikan parameter pada program yang dijalankan. Caranya adalah dengan menambahkan teks setelah nama \textit{file} utamanya, misalkan nama \textit{file} kode yang saya buat adalah \texttt{Test.kt}, setelah melalui proses \textit{compile}, menghasilkan \textit{file} dengan nama \texttt{TestKt.class}.

Cara eksekusi \textit{file} \texttt{TestKt.class} tersebut adalah dengan cara berikut :

\begin{lstlisting}
> kotlin TestKt [parameter]
\end{lstlisting}

Kali ini kita coba mengganti bagian \texttt{[parameter]} dengan teks \texttt{tamami}, perintah lengkapnya adalah sebagai berikut :

\begin{lstlisting}
> kotlin TestKt tamami
\end{lstlisting}

Hasil yang dikeluarkan dari pemanggilan kode program di atas adalah sebagai berikut :

\begin{lstlisting}
nama : tamami
\end{lstlisting}

Terlihat bahwa teks \texttt{tamami} yang menjadi parameter pertama, akan dilewatkan sebagai parameter di pemanggilan fungsi \texttt{cetak} seperti di baris ke-2, yang kemudian di cetak ke layar sebagaimana perintah pada baris ke-6.

\subsection{Cara Penggunaan}

\subsubsection{Notasi Infix}

Apa itu notasi \textit{infix}, notasi ini sebetulnya membuat pemanggilan sebuah fungsi menjadi seperti operator. Mungkin akan lebih mudah bila disajikan dalam bentuk kode program berikut :

\begin{lstlisting}
fun main(args: Array<String>) {
	var nama = "tamami"
	
	println(nama sambung "Selamat Datang")
}

infix fun String.sambung(data: String): String {
	return "$this $data"
}
\end{lstlisting}

Sehingga hasil yang muncul ke layar adalah sebagai berikut :

\begin{lstlisting}
tamami Selamat Datang
\end{lstlisting}

Perhatikan pada baris ke-4, kita melihat bahwa fungsi \texttt{sambung} yang dideklarasikan pada baris ke-7 sampai dengan baris ke-9, karena memiliki kata kunci \texttt{infix} dapat berbentuk seperti operator biasa, cara memanggilnya tidak seperti kondisi fungsi pada umumnya.

Pada baris ke-8, kita melihat ada 2 (dua) variabel yang dipanggil, yaitu \texttt{\$this} yang isinya adalah \texttt{String} yang akan disambung, dan variabel \texttt{\$data} yang isinya adalah \texttt{String} sambungannya.

Beberapa hal agar model \textit{infix} ini dapat diimplementasikan yaitu :

\begin{enumerate}
	\item Fungsi yang dideklarasikan merupakan anggota dari kelas (berada dalam sebuah kelas) atau merupakan fungsi ekstensi dari suatu kelas..
	\item Fungsi ini hanya memiliki sebuah parameter
	\item Fungsi ini harus dideklarasikan dengan kata kunci \texttt{infix}.
\end{enumerate}

\subsubsection{Parameter}

Parameter seperti halnya variabel atau properti, harus dideklarasikan dengan format \texttt{[nama] : [tipe data]}, nama pada kondisi deklarasi parameter wajib menyertakan tipe data secara eksplisit. Tiap parameter pada konstruktor atau fungsi dipisahkan dengan koma.

Coba perhatikan fungsi \texttt{sambung} pada kode sebelumnya, disana ada sebuah parameter dengan nama \texttt{data} dengan tipe data berupa \texttt{String}. Deklarasinya harus berbentuk seperti ini.

\subsubsection{Parameter \textit{Default}}

Sebuah fungsi dapat memiliki parameter yang secara \textit{default} sudah memiliki nilai. Parameter-parameter ini dapat dilewatkan saat memanggil fungsi. Contoh kodenya adalah sebagai berikut :

\begin{lstlisting}
fun main(args: Array<String>) {
	var pegawai = Pegawai()
	pegawai.setBio("tamami")
	
	println("nama : ${pegawai.nama}")
	println("jabatan : ${pegawai.jabatan}")
}

class Pegawai {
	var nama: String = ""
	var jabatan: String = ""
	
	fun setBio(nama: String, jabatan: String = "STAF") {
		this.nama = nama
		this.jabatan = jabatan
	}
}
\end{lstlisting}

Hasil keluaran dari kode di atas adalah sebagai berikut :

\begin{lstlisting}
nama : tamami
jabatan : STAF
\end{lstlisting}

Perhatikan pada baris ke-3 dari kode di atas, bahwa pemanggilan fungsi \texttt{setBio} milik kelas \texttt{Pegawai} hanya menyertakan sebuah parameter sedangkan pada deklarasinya membutuhkan 2 (dua) buah parameter, yaitu \texttt{nama} dan \texttt{jabatan}. 

Namun karena parameter \texttt{jabatan} memiliki nilai \textit{default} berupa teks \texttt{STAF}, maka parameter ini dapat dilewatkan, karena akan diisikan otomatis dengan teks \texttt{STAF}.

Maka dari itu hasil keluaran yang kita dapatkan adalah seperti tampilan di atas.

Apabila ada kelas turunan yang melakukan \textit{override} terhadap fungsi yang memiliki nilai \textit{default}, maka aturannya tidak boleh didefinisikan ulang nilai \textit{default}nya pada kelas turunan, jadi kelas turunannya akan mengikuti nilai \textit{default} milik pendahulunya.

Kita perhatikan contoh kode berikut :

\begin{lstlisting}
fun main(args: Array<String>) {
	var pegawai = Pejabat()
	pegawai.setBio("tamami")
	
	println("nama : ${pegawai.nama}")
	println("jabatan : ${pegawai.jabatan}")
}

open class Pegawai {
	var nama: String = ""
	var jabatan: String = ""
	
	open fun setBio(nama: String, jabatan: String = "STAF") {
		this.nama = nama
		this.jabatan = jabatan
	}
}

class Pejabat: Pegawai() {
	override fun setBio(nama: String, jabatan: String) {
		this.nama = nama
		if(jabatan.equals("STAF")) {
			println("Masa pejabat jabatannya staf")
			return
		}
		this.jabatan = jabatan
	}
}
\end{lstlisting}

Hasil keluaran untuk kode di atas adalah sebagai berikut :

\begin{lstlisting}
Masa pejabat jabatannya staf
nama : tamami
jabatan :
\end{lstlisting}

Lihatlah pada baris ke-19 sampai dengan baris ke-28, disini adalah tempat deklarasi kelas \texttt{Pejabat} yang sebetulnya adalah turunan dari kelas \texttt{Pegawai}, perhatikan kembali bahwa kelas \texttt{Pegawai} ini harus memiliki kata kunci \texttt{open} agar bisa diwariskan ke kelas di bawahnya, kemudian fungsi \texttt{setBio} milik kelas \texttt{Pegawai} juga diberikan kata kunci \texttt{open} agar kelas dibawahnya dapat melakukan \textit{override} terhadap fungsi ini.

Pada baris ke-20 fungsi \texttt{setBio} milik kelas \texttt{Pegawai} di \textit{override}, dimana pada parameter \texttt{jabatan} sudah tidak dapat lagi diberikan nilai \textit{default} karena nilai \textit{default} sudah diberikan pada kelas pendahulunya, yaitu kelas \texttt{Pegawai}.

Dan pada baris ke-22, kita lakukan seleksi, apabila nilai parameter \texttt{jabatan} masih berisi nilai \textit{default}, yaitu teks \texttt{STAF}, maka kita akan cetak informasi bahwa yang masuk kelas \texttt{Pejabat} ini tentunya harus sudah memberikan jabatan strukturan yang baru selain \texttt{STAF}, maka kemudian isi variabel / properti \texttt{jabatan} milik kelas \texttt{Pejabat} akan dikosongkan / tidak diisikan, sehingga hasil keluaran akan tampak seperti di atas.

\subsubsection{Argumen Bernama}

\subsubsection{Pengembalian Unit}

\subsubsection{Ekspresi Tunggal}

\subsubsection{Pengembalian Eksplisit}

\subsubsection{Argumen Dinamis}

\section{Fungsi Lanjutan dan Lamda}

\section{Fungsi Sebaris}

\section{\textit{Coroutines}}