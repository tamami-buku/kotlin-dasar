\chapter{Memulai} 

Perlu diketahui bahwa Kotlin ini adalah bahasa pemrograman yang berjalan di atas JVM, sehingga diperlukan Java Runtime untuk menjalankannya.

Cara termudah untuk memasangkan atau meng\textit{install} \textit{compiler} Kotlin adalah dengan mengunduh di halaman \url{https://github.com/JetBrains/kotlin/releases/}, kemudian melakukan \textit{unzip} dan menambahkan direktori \texttt{bin} ke dalam \textit{path} sistem.

\section{Sintak Dasar}

\subsection{Deklarasi Paket}

Sama seperti bahasa pemrograman Java, deklarasi paket berada di awal kode seperti contoh berikut :

\begin{lstlisting}
package nama.paket

import java.net.*
...
\end{lstlisting}

Perbedaannya adalah bahwa nama paket tidak perlu disesuaikan atau disamakan dengan nama direktorinya seperti pada pemrograman Java. \textit{File} kode sumber dapat ditempatkan dimanapun pada \textit{drive}.

\subsection{Deklarasi Fungsi}

Deklarasi fungsi tanpa parameter dan tanpa nilai balikkan (\textit{return}) akan terlihat seperti contoh kode berikut :

\begin{lstlisting}
fun cetak(): Unit {
	println("Hai, apa kabar")
}
\end{lstlisting}

Atau deklarasi \texttt{Unit} dapat dihilangkan dengan kode akan terlihat seperti ini :

\begin{lstlisting}
fun cetak() {
	println("Hai, apa kabar")
}
\end{lstlisting}

Untuk deklarasi fungsi dengan parameter akan terlihat seperti contoh kode berikut :

\begin{lstlisting}
fun tambah(a: Int, b: Int): Int {
	return a + b
}
\end{lstlisting}

Fungsi yang sama seperti diatas dapat dibuat lebih ringkas dengan nilai balikan \textit{return} yang sudah diprediksi oleh Kotlin, kodenya menjadi seperti berikut ini :

\begin{lstlisting}
fun tambah(a: Int, b: Int) = a + b
\end{lstlisting}

Untuk pembahasan lebih lanjut mengenai fungsi, akan dijabarkan dalam bagian tersendiri dalam buku ini.

\subsection{Deklarasi Variabel}

Deklarasi variabel dapat dilakukan untuk 2 (dua) cara. Yang pertama adalah variabel yang hanya dapat diisi satu kali, dan ada yang dapat diisi berkali-kali. 

Kode untuk deklarasi variabel yang hanya dapat diisi 1 (satu) kali adalah sebagai berikut :

\begin{lstlisting}
val a: Int = 2 
// atau 
val c = 2
// atau
val d: Int
d = 5
\end{lstlisting}

Untuk deklarasi variabel yang dapat diubah, kodenya adalah sebagai berikut :

\begin{lstlisting}
var e = 2
e *= 2
\end{lstlisting}

\subsection{Deklarasi Komentar}

Seperti bahasa pemrograman Java dan Javascript, Kotlin juga menyediakan komentar dalam bentuk komentar baris dan komentar multi-baris. Kode untuk komentar satu baris adalah sebagai berikut :

\begin{lstlisting}
// ini komentar 1 baris
\end{lstlisting}

Untuk kode komentar multi-bari adalah sebagai berikut :

\begin{lstlisting}
/*  ini komentar
    multi baris */
\end{lstlisting}

Namun tidak seperti bahasa pemrograman Java, komentar di Kotlin dapat bersarang bertingkat.

\section{Logat}

Beberapa logat yang biasa digunakan di Kotlin adalah seperti di bawah ini.

\subsection{Membuat Kelas Data}

Kelas data ini biasa digunakan untuk pembuatan kelas \textit{entity}. Contoh kodenya adalah sebagai berikut :

\begin{lstlisting}
data class Pegawai(val nim: String, val nama: String)
\end{lstlisting}

Dengan menambahkan deklarasi \texttt{data} di depan kelas, maka untuk kelas Pegawai ini akan disediakan fungsi-fungsi berikut secara otomatis :

\begin{itemize}
\item \textit{Getters} dan \textit{Setter} untuk seluruh properti
\item \textit{Method} \texttt{equals}.
\item \textit{Method} \texttt{hashCode}
\item \textit{Method} \texttt{toString}
\item \textit{Method} \texttt{copy}
\end{itemize}

\subsection{Nilai \textit{Default} Untuk Parameter Fungsi}

Pada saat deklarasi fungsi, sebetulnya parameter dapat kita isikan dengan nilai \textit{default} seperti berikut :

\begin{lstlisting}
fun isiData(nama: String, kelamin: Int = 0) {
	...
}
\end{lstlisting}

Nantinya parameter \texttt{kelamin} akan terisi otomatis dengan \texttt{0}

\subsection{\textit{Interpolasi} Teks}

\textit{Interpolasi} atau penyisipan teks akan terlihat seperti baris perintah berikut ini :

\begin{lstlisting}
val nama: String = "tamami"
println("name $nama")
\end{lstlisting}

Nantinya sisipan teks dengan kode \texttt{$nama} akan terisi oleh variabel \texttt{nama} yang telah dideklarasikan sebelumnya.

\subsection{Pemeriksaan Instan}
\subsection{Penggunaan \textit{Range}}
\subsection{\textit{Read-only List}}
\subsection{\textit{Read-only Map}}
\subsection{Mengakses \textit{Map}}
\subsection{Jalan Pintas Perintah \texttt{if not null}}
\subsection{Jalan Pintas Perintah \texttt{if not null and else}}
\subsection{Eksekusi Perintah \texttt{if null}}
\subsection{Eksekusi Perintah \texttt{if not null}}
\subsection{Kembalikan Pada Perintah \texttt{when}}
\subsection{Ekspresi \texttt{try catch}}
\subsection{Ekspresi \texttt{if}}

\section{Adat}