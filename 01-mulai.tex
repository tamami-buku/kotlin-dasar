\chapter{Memulai} 

Perlu diketahui bahwa Kotlin ini adalah bahasa pemrograman yang berjalan di atas JVM, sehingga diperlukan Java Runtime untuk menjalankannya.

Cara termudah untuk memasangkan atau meng\textit{install} \textit{compiler} Kotlin adalah dengan mengunduh di halaman \url{https://github.com/JetBrains/kotlin/releases/}, kemudian melakukan \textit{unzip} dan menambahkan direktori \texttt{bin} ke dalam \textit{path} sistem.

Untuk memastikan bahwa Kotlin sudah terpasang dan dapat digunakan, kita seharusnya dapat menjalankan perintah berikut di konsol pada Linux atau \textit{command prompt} milik Windows, berikut perintahnya :

\begin{lstlisting}
kotlinc -version
\end{lstlisting}

Perintah tersebut sebetulnya untuk mencetak informasi tentang versi \textit{compiler} Kotlin yang aktif. Dan seharusnya akan muncul informasi yang kurang lebih sebagai berikut :

\begin{lstlisting}
info: Kotlin Compiler version 1.1.2-2
\end{lstlisting}

Tentunya versi yang keluar akan berbeda tergantung apa yang kita \textit{install}.

Percobaan berikutnya adalah menampilkan versi \textit{runtime environment} dari Kotlin, jika perintah \texttt{kotlinc} digunakan untuk melakukan \textit{compile} (kompilasi) terhadap kode yang kita ketik / tulis menjadi bahasa biner, fungsi dari \textit{runtime environment} adalah menerjemahkan bahasa biner hasil \textit{compile} oleh \texttt{kotlinc} menjadi bahasa \textit{native} sesuai sistem operasi yang digunakan, inilah prinsip yang digunakan bahasa pemrograman Java yang tetap digunakan oleh Kotlin, karena memang Kotlin masih menggunakan JRE (\textit{Java Runtime Environment}).

Perintah untuk melihat versi \textit{runtime environment} dari Kotlin adalah sebagai berikut :

\begin{lstlisting}
kotlin -version
\end{lstlisting}

Dengan hasil keluaran di layar monitor seperti ini :

\begin{lstlisting}
Kotlin version 1.1.2-2 (JRE 1.8.0_121-b13)
\end{lstlisting}

Versi Kotlin seharusnya sama dengan versi \textit{compiler}-nya. Sedangkan muncul tambahan informasi \texttt{JRE 1.8.0\_121-b13}, inilah yang menunjukan bahwa Kotlin masih menggunakan JRE untuk menjalankan programnya, karena memang sebelum melakukan instalasi Kotlin, Java harus di\textit{install} terlebih dahulu.

\section{Katakan Hai}

Setelah melakukan percobaan dasar seperti di atas, kita akan mencoba menjalankan kode pertama yang kita buat dengan Kotlin. Berikut adalah langkahnya :

\begin{enumerate}[1.]
	\item Membuka editor teks seperti notepad, atom, notepad++, atau aplikasi sejenis. 
	\item Mengetikan kode berikut :
	
\begin{lstlisting}
fun main(args: Array<String>) {
    println("Hai, selamat datang")
}
\end{lstlisting}

	\item Simpanlah dengan nama apapun, berikan ekstensi \texttt{kt}, misal kita beri nama \textit{file} tersebut dengan \texttt{Test.kt}.
	\item Buka konsol atau \textit{command prompt} dan aktifkan ke direktori tempat kita simpan \textit{file} \texttt{Test.kt} tadi.
	\item \textit{Compile file} \texttt{Test.kt} tersebut dengan perintah berikut :
	
\begin{lstlisting}
kotlinc Test.kt
\end{lstlisting}

	\item Hasil dari \textit{compile} tersebut adalah berupa \textit{file} \texttt{TestKt.class}
	\item Untuk menjalankan hasil program yang telah kita \textit{compile}, gunakan perintah berikut :
	
\begin{lstlisting}
kotlin TestKt
\end{lstlisting}

	\item Kemudian akan program / aplikasi akan menghasilkan keluaran sebagai berikut :
	
\begin{lstlisting}
Hai, selamat datang
\end{lstlisting}

	\item Sampai titik ini, kita berhasil menjalankan kode yang telah kita buat.
\end{enumerate}

Jadi sebetulnya, untuk memulai koding dengan bahasa Kotlin cukup sederhana, tinggal siapkan \textit{berekstensi} \texttt{kt}, kemudian sertakan blok kode program berikut :

\begin{lstlisting}
fun main(args: Array<String>) {
	...
}
\end{lstlisting}

Seluruh program yang dibangun dengan Kotlin akan berawal dari fungsi \texttt{main} ini.

\section{Sintak Dasar}

\subsection{Deklarasi Paket}

Sama seperti bahasa pemrograman Java, deklarasi paket berada di awal kode seperti contoh berikut :

\begin{lstlisting}
package nama.paket

import java.net.*
...
\end{lstlisting}

Perbedaannya adalah bahwa nama paket tidak perlu disesuaikan atau disamakan dengan nama direktorinya seperti pada pemrograman Java. \textit{File} kode sumber dapat ditempatkan dimanapun pada \textit{drive}.

\subsection{Deklarasi Fungsi}

Deklarasi fungsi tanpa parameter dan tanpa nilai balikkan (\textit{return}) akan terlihat seperti contoh kode berikut :

\begin{lstlisting}
fun cetak(): Unit {
	println("Hai, apa kabar")
}
\end{lstlisting}

Atau deklarasi \texttt{Unit} dapat dihilangkan dengan kode akan terlihat seperti ini :

\begin{lstlisting}
fun cetak() {
	println("Hai, apa kabar")
}
\end{lstlisting}

Untuk deklarasi fungsi dengan parameter akan terlihat seperti contoh kode berikut :

\begin{lstlisting}
fun tambah(a: Int, b: Int): Int {
	return a + b
}
\end{lstlisting}

Fungsi yang sama seperti diatas dapat dibuat lebih ringkas dengan nilai balikan \textit{return} yang sudah diprediksi oleh Kotlin, kodenya menjadi seperti berikut ini :

\begin{lstlisting}
fun tambah(a: Int, b: Int) = a + b
\end{lstlisting}

Untuk pembahasan lebih lanjut mengenai fungsi, akan dijabarkan dalam bagian tersendiri dalam buku ini.

\subsection{Deklarasi Variabel}

Deklarasi variabel dapat dilakukan untuk 2 (dua) cara. Yang pertama adalah variabel yang hanya dapat diisi satu kali, dan ada yang dapat diisi berkali-kali. 

Kode untuk deklarasi variabel yang hanya dapat diisi 1 (satu) kali adalah sebagai berikut :

\begin{lstlisting}
val a: Int = 2 
// atau 
val c = 2
// atau
val d: Int
d = 5
\end{lstlisting}

Untuk deklarasi variabel yang dapat diubah, kodenya adalah sebagai berikut :

\begin{lstlisting}
var e = 2
e *= 2
\end{lstlisting}

\subsection{Deklarasi Komentar}

Seperti bahasa pemrograman Java dan Javascript, Kotlin juga menyediakan komentar dalam bentuk komentar baris dan komentar multi-baris. Kode untuk komentar satu baris adalah sebagai berikut :

\begin{lstlisting}
// ini komentar 1 baris
\end{lstlisting}

Untuk kode komentar multi-bari adalah sebagai berikut :

\begin{lstlisting}
/*  ini komentar
    multi baris */
\end{lstlisting}

Namun tidak seperti bahasa pemrograman Java, komentar di Kotlin dapat bersarang bertingkat.

\section{Logat}

Beberapa logat yang biasa digunakan di Kotlin adalah seperti di bawah ini.

\subsection{Membuat Kelas Data}

Kelas data ini biasa digunakan untuk pembuatan kelas \textit{entity}. Contoh kodenya adalah sebagai berikut :

\begin{lstlisting}
data class Pegawai(val nim: String, val nama: String)
\end{lstlisting}

Dengan menambahkan deklarasi \texttt{data} di depan kelas, maka untuk kelas Pegawai ini akan disediakan fungsi-fungsi berikut secara otomatis :

\begin{itemize}
\item \textit{Getters} dan \textit{Setter} untuk seluruh properti
\item \textit{Method} \texttt{equals}.
\item \textit{Method} \texttt{hashCode}
\item \textit{Method} \texttt{toString}
\item \textit{Method} \texttt{copy}
\end{itemize}

\subsection{Nilai \textit{Default} Untuk Parameter Fungsi}

Pada saat deklarasi fungsi, sebetulnya parameter dapat kita isikan dengan nilai \textit{default} seperti berikut :

\begin{lstlisting}
fun isiData(nama: String, kelamin: Int = 0) {
	...
}
\end{lstlisting}

Nantinya parameter \texttt{kelamin} akan terisi otomatis dengan \texttt{0}

Contoh kodenya adalah sebagai berikut :

\begin{lstlisting}
fun main(args: Array<String>) {
	val nama = "tamami"
	println("Halo, $nama")
	
	isiData(nama)
}

fun isiData(nama: String, kelamin: Int = 0) {
	println(kelamin)
}
\end{lstlisting}

Hasil keluarannya adalah sebagai berikut :

\begin{lstlisting}
Halo, tamami
0
\end{lstlisting}

Penjelasannya adalah sebagai berikut, pada baris pertama menghasilkan keluaran teks \texttt{Halo, tamami}, yang sebetulnya hasil dari eksekusi perintah kode pada baris ke-3, yaitu :

\begin{lstlisting}
println("Halo, $nama")
\end{lstlisting}

Dimana pemanggilan variabel \texttt{\$nama} pada baris ke-2 dari \textit{source code} terjadi, dan yang ditampilkan di layar monitor adalah isi dari variabel \texttt{\$nama}, yaitu \texttt{tamami}.

Sedangkan pada baris kedua dari hasil keluaran, yaitu \texttt{0}, adalah hasil dari eksekusi kode pada bari ke-9, di dalam fungsi \texttt{isiData}, tepatnya pada perintah berikut :

\begin{lstlisting}
println(kelamin)
\end{lstlisting}

Kenapa hasil keluarannya adalah \texttt{0}, alurnya adalah seperti ini, pada saat pemanggilan fungsi \texttt{isiData(nama)} pada baris ke-5, parameter \texttt{nama} pada fungsi \texttt{isiData} ini terisi dengan nilai \texttt{tamami}, karena parameter kedua, yaitu \texttt{kelamin} tidak disertakan pada pemanggilannya pada baris ke-5, sehingga parameter \texttt{kelamin} akan terisi otomatis dengan nilai \texttt{0} sebagaimana deklarasinya pada baris ke-8.

\subsection{\textit{Interpolasi} Teks}

\textit{Interpolasi} atau penyisipan teks akan terlihat seperti baris perintah berikut ini :

\begin{lstlisting}
val nama: String = "tamami"
println("name $nama")
\end{lstlisting}

Nantinya sisipan teks dengan kode \texttt{\$nama} akan terisi oleh variabel \texttt{nama} yang telah dideklarasikan sebelumnya.

\subsection{Pemeriksaan Instan}

Pada bahasa Kotlin, kita dapat melakukan pemeriksaan tipe data secara instan, formatnya adalah seperti kode berikut :

\begin{lstlisting}
when(x) {
	is String -> ...
	is Int -> ...
	is KelasSaya -> ...
	else -> ...
}
\end{lstlisting}

Artinya nanti isi dari variabel \texttt{x} akan dipilah, apakah merupakan tipe data \texttt{String}, \texttt{Int}, merupakan instan dari kelas \texttt{KelasSaya}, atau berupa tipe data atau kelas lain.

Contoh nyata dari penggunaan kode di atas adalah sebagai berikut :

\begin{lstlisting}
fun main(args: Array<String>) {
	val x: Any = 2
	
	when (x) {
		is String -> println("Jawaban String")
		is Int -> println("Jawaban Int")
		else -> println("lainnya")
	}
}
\end{lstlisting}

Pada kode di atas, tipe data dari variabel \texttt{x} adalah \texttt{Any}, yang artinya bisa berupa tipe data apapun, atau instan dari kelas apapun. Lalu diisikan nilai awal berupa angka 2 (dua).

Selanjutnya kode akan melakukan seleksi tipe data \texttt{x} pada baris ke-4 dengan perintah \texttt{when (x)}, kemudian melakukan pemeriksaan, apabila tipe data dari variabel \texttt{x} adalah \texttt{String} maka akan dicetak seperti pada baris ke-5 dari kode di atas, tapi ternyata memang tipe data yang tepat adalah pada baris ke-6 sehingga program yang kita bangun akan mencetak \texttt{Jawaban Int} di layar, karena variabel \texttt{x} berisi angka 2 (dua).

\subsection{Penggunaan \textit{Range}}

Penggunaan \textit{range} biasanya untuk melakukan iterasi atau perulangan, beberapa contohnya akan dikerjakan dengan kode berikut :

\begin{lstlisting}
for(i in 1..10) { println("data ke-$i = $i") }
\end{lstlisting}

Kode tersebut akan menghasilkan keluaran di monitor dimana nilai \texttt{i} dari 1 sampai dengan 10 sebagai berikut :

\begin{lstlisting}
data ke-1 = 1
data ke-2 = 2
data ke-3 = 3
data ke-4 = 4
data ke-5 = 5
data ke-6 = 6
data ke-7 = 7
data ke-8 = 8
data ke-9 = 9
data ke-10 = 10
\end{lstlisting}

Contoh penggunaan \textit{range} yang lain adalah seperti kode berikut :

\begin{lstlisting}
for(i in 1 until 10) { println("data ke-$i = $i") }
\end{lstlisting}

Sama seperti kode sebelumnya, hanya saja kali ini nilai \texttt{i} adalah antara 1 sampai dengan 9, angka 10 tidak masuk dalam kualifikasi proses cetak ke monitor. Berikut adalah hasil keluarannya di monitor :

\begin{lstlisting}
data ke-1 = 1
data ke-2 = 2
data ke-3 = 3
data ke-4 = 4
data ke-5 = 5
data ke-6 = 6
data ke-7 = 7
data ke-8 = 8
data ke-9 = 9
\end{lstlisting}

Contoh penggunaan \textit{range} dengan beberapa pola lompatan atau kelipatan angka adalah sebagai berikut :

\begin{lstlisting}
for(i in 1..10 step 3) { println("data ke-$i") }
\end{lstlisting}

Maksudnya adalah mencetak deretan angka yang dimulai dari 1 dengan kelipatan 3 sampai nilai \texttt{i} sama dengan 10. Berikut adalah hasil keluaran dari kode tersebut :

\begin{lstlisting}
data ke-1
data ke-4
data ke-7
data ke-10
\end{lstlisting}

Contoh penggunaan \textit{range} untuk perulangan yang mundur ke nilai yang lebih kecil adalah sebagai berikut :

\begin{lstlisting}
for(i in 10 downTo 1) { println("data ke-$i") }
\end{lstlisting}

Penjelasan dari kode tersebut dijelaskan dengan hasil keluaran di layar monitor sebagai berikut :

\begin{lstlisting}
data ke-10
data ke-9
data ke-8
data ke-7
data ke-6
data ke-5
data ke-4
data ke-3
data ke-2
data ke-1
\end{lstlisting}

\subsection{\textit{Read-only List}}

Artinya adalah membuat \textit{list} yang tidak dapat diubah isinya, contoh deklarasinya adalah sebagai berikut :

\begin{lstlisting}
val list = listOf("data A", "data B", "data C")
\end{lstlisting}

Untuk lebih jelasnya, kita akan melihat kode contoh sebagai berikut :

\begin{lstlisting}
fun main(args: Array<String>) {
	val list = listOf("A", "B", "C");
	
	for(i in list) {
		println("data $i")	
	]
}
\end{lstlisting}

Kode tersebut, pada baris ke-2 adalah menyiapkan objek \textit{list} dan diisikan langsung dengan data menggunakan perintah \texttt{listOf}, selanjutnya seluruh data dicetak ke layar monitor sebagaimana tampilan berikut :

\begin{lstlisting}
data A
data B
data C
\end{lstlisting}

\subsection{\textit{Read-only Map}}
\subsection{Mengakses \textit{Map}}
\subsection{Jalan Pintas Perintah \texttt{if not null}}
\subsection{Jalan Pintas Perintah \texttt{if not null and else}}
\subsection{Eksekusi Perintah \texttt{if null}}
\subsection{Eksekusi Perintah \texttt{if not null}}
\subsection{Kembalikan Pada Perintah \texttt{when}}
\subsection{Ekspresi \texttt{try catch}}
\subsection{Ekspresi \texttt{if}}

\section{Adat}