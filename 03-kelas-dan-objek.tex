\chapter{Kelas dan Objek}

\section{Kelas}

Kelas di Kotlin dideklarasikan dengan kata kunci \texttt{class}. Contoh dalam kodenya adalah sebagai berikut :

\begin{lstlisting}
class Mahasiswa {
}
\end{lstlisting}

Deklarasi kelas memang sesederhana itu. Isi dari kelas itu sendiri terdiri dari konstruktor, properti atau yang biasa dikenal dengan istilah variabel, dan fungsi.

Uniknya bentuk konstruktor dari Kotlin ini dibedakan menjadi 2 (dua), ada konstruktor utama, dan ada konstruktor tambahan. Kita bahas terlebih dahulu bagaimana bentuk dari konstruktor utama, contoh kode dasarnya adalah sebagai berikut :

\begin{lstlisting}
class Mahasiswa constructor(nama: String) {
}
\end{lstlisting}

Jika konstruktor tidak memiliki anotasi atau \textit{visibility modifiers}, maka kode di atas dapat disederhanakan menjadi seperti berikut :

\begin{lstlisting}
class Mahasiswa(nama: String) {
}
\end{lstlisting}

Tentang apa itu anotasi dan \textit{visibility modifiers} akan kita jelaskan di bab berikutnya. 

Lalu bagaimana cara memanfaatkan parameter yang ada pada konstruktor bila deklarasi konstruktor implisit seperti itu? Ada 2 (dua) cara, yang pertama melalui blok \texttt{init}, yang kedua dengan langsung mengisikan ke variabel yang bersangkutan. 

Berikut adalah contoh dari penggunaan blok \texttt{init} :

\begin{lstlisting}
fun main(args: Array<String>) {
	val mhs = Mahasiswa("tamami")
	
	println(mhs.nama)
}

class Mahasiswa(nama: String) {
	var nama: String
	
	init {
		this.nama = nama
	}
}
\end{lstlisting}

Pada baris ke-2 dari kode di atas, variabel \texttt{mhs} bertipe kelas \texttt{Mahasiswa} yang langsung dipanggil konstruktornya dengan parameter berupa teks (\textit{string}). 

Kotlin akan memanggil konstruktor utamanya kemudian menjalankan blok \texttt{init} untuk pertama kalinya. Isi dari blok \texttt{init} ini hanya mengisikan variabel \texttt{nama} dari parameter konstruktor \texttt{nama}.

Kemudian aplikasi menlanjutkan tugasnya untuk mencetak variabel \texttt{nama} milik instan kelas \texttt{Mahasiswa}.

Atau kita juga bisa persingkat kode di atas menjadi seperti berikut :

\begin{lstlisting}
fun main(args: Array<String>) {
	val mhs = Mahasiswa("tamami")
	
	println(mhs.nama)
}

class Mahasiswa(nama: String) {
	var nama = nama
}
\end{lstlisting}

Hasil keluaran dari kode di atas sama persis dengan sebelumnya, seperti ini :

\begin{lstlisting}
tamami
\end{lstlisting}

Kita telah menghapus blok \texttt{init} dan melewatkan nilai dari parameter \texttt{nama} langsung ke variabel \texttt{nama}.

Lalu bagaimana dengan konstruktor tambahan, deklarasi nya ada di dalam tubuh kelas itu sendiri, contoh kodenya adalah seperti berikut :

\begin{lstlisting}
fun main(args: Array<String>) {
	val mhs = Mahasiswa("tamami", "DIV-TI")
	
	println(mhs.nama)
	println(mhs.jurusan)
}

class Mahasiswa(nama: String) {
	var nama = nama
	var jurusan: String = ""
	
	constructor(nama: String, jurusan: String): this(nama) {
		this.jurusan = jurusan
	}
}
\end{lstlisting}

Deklarasi konstruktor utama ada di baris ke-8 dengan satu parameter yaitu \texttt{nama}, sedangkan deklarasi konstruktor tambahan ada pada baris ke-12 sampai dengan baris ke-14. Dimana konstruktor tambahan memiliki 2 (dua) parameter, yaitu \texttt{nama} dan \texttt{jurusan}. 

Ada satu tambahan lagi pada konstruktor tambahan, yaitu perintah \texttt{this} di akhir baris, ini karena Kotlin mengharuskan seluruh konstruktor tambahan memanggil konstruktor utama terlebih dahulu dengan perintah \texttt{this}.

Dalam sebuah kelas dapat memuat beberapa hal berikut :

\begin{itemize}
	\item Konstruktor dan blok \texttt{init}
	\item Fungsi
	\item Properti (atau lebih dikenal dengan variabel)
	\item Kelas bersarang
	\item Deklarasi Objek.
\end{itemize}

Sebetulnya seluruh kelas di Kotlin akan bermuara pada kelas \texttt{Any} sebagai super-kelas-nya. Bahkan kelas-kelas yang deklarasinya tanpa super-kelas akan menjadikan kelas \texttt{Any} ini sebagai \textit{default}.

Untuk mendeklarasikan super kelas secara eksplisit, contoh kode berikut akan menjelaskannya :

\begin{lstlisting}
fun main(args: Array<String>) {
	val pegawai = Pejabat("tamami", "fungsional")
	
	println(pegawai.nama)
	println(pegawai.jabatan)
}

open class Pegawai(nama: String) {
	var nama = nama
}

class Pejabat(nama: String, jabatan: String): Pegawai(nama) {
	var jabatan = jabatan
}
\end{lstlisting}

Pada kode di atas, yang menerangkan deklarasi super kelas secara eksplisit tepat pada baris ke-12, yang menunjukkan bahwa kelas \texttt{Pejabat} yang dibentuk adalah turunan dari kelas \texttt{Pegawai}.

Pada deklarasi kelas \texttt{Pegawai}, ada pernyataan \texttt{open} disana, ini sebetulnya menandakan bahwa kelas tersebut bukan bersifat final, karena secara \textit{default}, semua kelas yang dibentuk di Kotlin akan bersifat final, maka agar kita dapat membuat turunan dari kelas yang telah kita buat, maka kelas tersebut harus kita berikan tanda \texttt{open} di awal deklarasi kelas.

Alur dari kode program di atas dapat diceritakan demikian, pertama pada baris ke-2 kita membuat sebuah variabel bernama \texttt{pegawai}, kemudian diisikan dengan data dari instan kelas \texttt{Pejabat}.

Kita coba melompat ke baris 12, dimana ini adalah tempat deklarasi pembentukan kelas \texttt{Pejabat} yang memang memiliki 2 (dua) parameter. Namun pada baris ke-12 inilah secara eksplisit menyebutkan bahwa kelas \texttt{Pejabat} adalah turunan dari kelas \texttt{Pegawai}. Namun parameter yang dimasukan ke kelas \texttt{Pegawai} adalah parameter yang juga masuk melalui konstruktor \texttt{Pejabat}, sehingga parameter yang dilewatkan ke konstruktor \texttt{Pegawai} adalah parameter yang juga dibawah oleh konstruktor \texttt{Pejabat}.

Pemanggilan variabel atau properti dari kelas \texttt{Pejabat} di baris ke-4 dan ke-5 sebetulnya tidak aneh karena sebetulnya, setelah kelas \texttt{Pejabat} menjadi turunan dari kelas \texttt{Pegawai}, maka semua variabel dan fungsi yang ada pada kelas \texttt{Pegawai} akan dimiliki oleh kelas \texttt{Pejabat}.

\section{Properti}

\section{\textit{Interface}}

\section{\textit{Visibility Modifiers}}

\section{Ekstensi}

\section{Kelas Data}

\section{Kelas Tertutup}

\section{Generik}

\section{Kelas Bersarang}

\section{Kelas \textit{Enum}}

\section{Ekspresi Objek dan Deklarasi}

\section{Delegasi}

\section{Mendelegasikan Properti}