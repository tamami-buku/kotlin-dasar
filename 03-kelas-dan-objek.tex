\chapter{Kelas dan Objek}

\section{Kelas}

Kelas di Kotlin dideklarasikan dengan kata kunci \texttt{class}. Contoh dalam kodenya adalah sebagai berikut :

\begin{lstlisting}
class Mahasiswa {
}
\end{lstlisting}

Deklarasi kelas memang sesederhana itu. Isi dari kelas itu sendiri terdiri dari konstruktor, properti atau yang biasa dikenal dengan istilah variabel, dan fungsi.

Uniknya bentuk konstruktor dari Kotlin ini dibedakan menjadi 2 (dua), ada konstruktor utama, dan ada konstruktor tambahan. Kita bahas terlebih dahulu bagaimana bentuk dari konstruktor utama, contoh kode dasarnya adalah sebagai berikut :

\begin{lstlisting}
class Mahasiswa constructor(nama: String) {
}
\end{lstlisting}

Jika konstruktor tidak memiliki anotasi atau \textit{visibility modifiers}, maka kode di atas dapat disederhanakan menjadi seperti berikut :

\begin{lstlisting}
class Mahasiswa(nama: String) {
}
\end{lstlisting}

Tentang apa itu anotasi dan \textit{visibility modifiers} akan kita jelaskan di bab berikutnya. 

Lalu bagaimana cara memanfaatkan parameter yang ada pada konstruktor bila deklarasi konstruktor implisit seperti itu? Ada 2 (dua) cara, yang pertama melalui blok \texttt{init}, yang kedua dengan langsung mengisikan ke variabel yang bersangkutan. 

Berikut adalah contoh dari penggunaan blok \texttt{init} :

\begin{lstlisting}
fun main(args: Array<String>) {
	val mhs = Mahasiswa("tamami")
	
	println(mhs.nama)
}

class Mahasiswa(nama: String) {
	var nama: String
	
	init {
		this.nama = nama
	}
}
\end{lstlisting}

Pada baris ke-2 dari kode di atas, variabel \texttt{mhs} bertipe kelas \texttt{Mahasiswa}

\section{Properti}

\section{\textit{Interface}}

\section{\textit{Visibility Modifiers}}

\section{Ekstensi}

\section{Kelas Data}

\section{Kelas Tertutup}

\section{Generik}

\section{Kelas Bersarang}

\section{Kelas \textit{Enum}}

\section{Ekspresi Objek dan Deklarasi}

\section{Delegasi}

\section{Mendelegasikan Properti}